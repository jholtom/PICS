\documentclass{hitec}

\author{Andrew Wygle}
\title{Elysium Radio Data Link Layer - Space Data Link Protocols}
\company{Adamant Aerospace}
\confidential{\textbf{Unlimited Distribution}}

\usepackage{graphicx}
\usepackage{booktabs}
\usepackage{pdfpages}
\usepackage{gensymb}
\usepackage{siunitx}
\usepackage{bytefield}
\usepackage[hyperref=true]{register}
\usepackage{rotating}
\usepackage{titlesec}
\usepackage{subcaption}
\usepackage{fmtcount}
\usepackage{parskip}
\usepackage{caption}
\usepackage[autostyle, english = american]{csquotes}
\usepackage{enumitem}
\usepackage{longtable}
\usepackage{array}
\usepackage[colorlinks=true, linktoc=all]{hyperref}
\usepackage{etoolbox}

\MakeOuterQuote{"}

\settextfraction{1}

\newcommand{\chref}[1]{\hyperref[ch:#1]{Section~\ref{ch:#1}}}
\newcommand{\figref}[1]{\hyperref[fig:#1]{Figure~\ref{fig:#1}}}
\newcommand{\tabref}[1]{\hyperref[tab:#1]{Table~\ref{tab:#1}}}
\makeatletter
\newcommand*{\textoverline}[1]{$\overline{\hbox{#1}}\m@th$}
\makeatother

\newcommand{\bitlabel}[2]{%
	\bitbox[]{#1}{%
		\raisebox{0pt}[4ex][0pt]{%
			\turnbox{45}{\fontsize{7}{7}\selectfont#2}%
		}%
	}%
}

\titleclass{\subsubsubsection}{straight}[\subsection]

\newcounter{subsubsubsection}[subsubsection]
\renewcommand\thesubsubsubsection{\thesubsubsection.\arabic{subsubsubsection}}
\renewcommand\theparagraph{\thesubsubsubsection.\arabic{paragraph}} % optional; useful if paragraphs are to be numbered

\titleformat{\subsubsubsection}
  {\normalfont\normalsize\bfseries}{\thesubsubsubsection}{1em}{}
\titlespacing*{\subsubsubsection}
{0pt}{3.25ex plus 1ex minus .2ex}{1.5ex plus .2ex}

\makeatletter
\renewcommand\paragraph{\@startsection{paragraph}{5}{\z@}%
  {3.25ex \@plus1ex \@minus.2ex}%
  {-1em}%
  {\normalfont\normalsize\bfseries}}
\renewcommand\subparagraph{\@startsection{subparagraph}{6}{\parindent}%
  {3.25ex \@plus1ex \@minus .2ex}%
  {-1em}%
  {\normalfont\normalsize\bfseries}}
\def\toclevel@subsubsubsection{4}
\def\toclevel@paragraph{5}
\def\toclevel@paragraph{6}
\def\l@subsubsubsection{\@dottedtocline{4}{7em}{4em}}
\def\l@paragraph{\@dottedtocline{5}{10em}{5em}}
\def\l@subparagraph{\@dottedtocline{6}{14em}{6em}}
\makeatother

\setcounter{secnumdepth}{4}
\setcounter{tocdepth}{4}

\titleformat{\subsubsection}
	{\normalfont\fontsize{12}{15}\bfseries}{\thesubsubsection}{1em}{}


\newcounter{regaddr}
\newcounter{regsize}
\newcommand*{\elyregaddr}[1][1]{0x\padzeroes[2]\Hexadecimal{regaddr}\addtocounter{regaddr}{#1}\setcounter{regsize}{#1}}

\newcommand*{\curreg}[1][0]{%
		\addtocounter{regaddr}{\the\numexpr#1-\value{regsize}\relax}%
		0x\padzeroes[2]\Hexadecimal{regaddr}%
		\addtocounter{regaddr}{\the\numexpr\value{regsize}-#1\relax}%
}

\newcounter{idval}
\newcommand*{\elyid}[0]{0x\padzeroes[2]\Hexadecimal{idval}\addtocounter{idval}{1}}

\newcommand*{\currid}[0]{%
		\addtocounter{idval}{-1}%
		0x\padzeroes[2]\Hexadecimal{idval}%
		\addtocounter{idval}{1}%
}

\newtoggle{inTableHeader}
\toggletrue{inTableHeader}
\newcommand*{\StartTableHeader}{\global\toggletrue{inTableHeader}}
\newcommand*{\EndTableHeader}{\global\togglefalse{inTableHeader}}
\let\OldLongtable\longtable%
\let\OldEndLongtable\endlongtable%
\renewenvironment{longtable}{\StartTableHeader\OldLongtable}{\OldEndLongtable\StartTableHeader}%

\newcommand*{\setlabref}[1]{\gdef\currlabref{#1}}
\newcommand*{\labref}[1]{\iftoggle{inTableHeader}{}{\hyperref[\currlabref]{#1}}}
\newcolumntype{W}{>{\global\let\currlabref\relax}l<{\labref{\elyid}}}
\newcolumntype{R}{>{\global\let\currlabref\relax}l<{\labref{\elyregaddr}}}

\let\oldaddcontentsline\addcontentsline
\newcommand{\stoptocentries}{\renewcommand{\addcontentsline}[3]{}}
\newcommand{\starttocentries}{\let\addcontentsline\oldaddcontentsline}

\begin{document}

% NOTE page numbering is controlled by scrreprt for the initial stuff - look into this later

\maketitle

\tableofcontents
\listoffigures
\listoftables

% Upon testing, we're going to use Chapters and Sections but not Parts
\section{Overview}
\label{ch:overview}

The Space Data Link Protocols (SDLP) are a group of CCSDS Recommended Standards
defining the Data Link Layer services of a space link. They are used by many
NASA and ESA missions. Four different SDLP standards exist - TM (Telemetry), TC
(Telecommand), AOS (Advanced Orbiting Systems), and Prox-1 (Proximity-1 Space
Link Protocol). The Elysium radio supports the TM and TC protocols, using TM
for downlink (transmission) and TC for uplink (reception).

The Elysium does not currently support the Space data Link Security Protocol.
If your mission requires the SDLS protocol, please contact Adamant to discuss
adding suport.

Registers associated with the SDLP Data Link Layer subsystem can be found in
\hyperref[sec:sdlpregs]{Section \ref{sec:sdlpregs}}. Channels, Errors, and
Events associated with the SDLP Data Link Layer subsystem can be found in
\hyperref[sec:sdlpchan]{Section \ref{sec:sdlpchan}},
\hyperref[sec:sdlperrs]{Section \ref{sec:sdlperrs}}, and
\hyperref[sec:sdlpevts]{Section \ref{sec:sdlpevts}}, respectively.


\subsection{TM Protocol}

The TM protocol makes use of either one Master Channel or one or more
multiplexed Virtual Channels to transfer fixed-length Transfer Frames over a
Physical Channel. A Synchronization and Channel Coding Sublayer optionally
encodes the TF using one of several types of FEC. 

The Elysium makes use of 2 independent Virtual Channels, referred to as VC0 and
VC1. Lower numbered VCs are treated as higher-priority channels. "Live" data
(data coming in over the UART interface) uses VC0, while "logged" data being
read out of on board storage uses VC1.

When used with the Space Packet Protocol Networking Layer, the Virtual Channel
Packet service is used to transfer the Live packets over Virtual Channel 0.
When used with other Networking Layer protocols, either the Virtual Channel
Packet service with the Encapsulation packet standard or the Virtual Channel
Access service may be used. The Virtual Channel Access service is always used
to transfer "logged" telemetry on VC1. See the Standard for more details.

TM Transfer Frames are transmitted sequentially in order and without gaps. The
timing of release is "mission-specific" according to the standard. The Elysium
takes the approach that Transfer Frames are sent as quickly as possible for as
long as there is data available to be sent.  When all available data has been
sent, the radio will cease transmission to conserve power. When data is being
downlinked due to a GetTelem command, a single OID
frame will be sent at the end of the Duration specified in the GetTelem command
to signify the end of telemetry data.

OID frames will also be generated to serve as beacon frames in certain fault
scenarios. 

The TM Synchronization and Channel Coding Sublayer standard
(\href{https://public.ccsds.org/Pubs/131x0b2ec1.pdf}{CCSDS Recommended Standard
131.0-B-2}) defines a large number of FEC encodings for use with TM space
links. At present, only the rate 1/2 convolutional coding is supported.
However, additional encodings are constantly under development. If your mission
requires a particular FEC encoding, please contact Adamant to discuss adding
support for it.

The data randomizer specified in the Synchronization and Channel Coding
Sublayer is also optionally supported.

\subsection{TC Protocol}

The TC protocol makes use of either one Master Channel or one or more
multiplexed Virtual Channels to transfer variable-length Transfer Frames over a
Physical Channel. A Synchronization and Channel Coding Sublayer optionally
encodes the TF using a modified BCH code for either error detection or error
correction.

The TC protocol also specifies the optional use of Multiplexer Access Points
(MAPs). The Elysium does not support the use of MAPs.

The Elysium makes use of a single Virtual Channel, referred to as VC0, for
uplink data under the TC protocol. 

When used with the Space Packet Protocol Networking Layer, the Virtual Channel
Packet service is used to transfer the packets over the Virtual Channels. When
used with other Networking Layer protocols, either the Virtual Channel Packet
service with the Encapsulation packet standard or the Virtual Channel Access
service may be used. See the Standard for more details.

The TC standard also makes reference to the COP-1 standard
(\href{https://public.ccsds.org/Pubs/232x1b2.pdf}{CCSDS 232.1-B-2}) which
provides an ARQ functionality for TC frames by using the OCF field of the
returning TM packets to serve as an acknowledgement function. The Frame
Acceptance and Reporting Mechanism from COP-1 is optionally available on the
Elyisum radio. The returned information occupies 8 bytes of each TM transfer
frame. See the Standards for more details.

The TC Synchronization and Channel Coding Sublayer standard
(\href{https://public.ccsds.org/Pubs/231x0b2c1.pdf}{CCSDS 231.0-B-2}) specifies
a modified BCH code for either error detection or error correction. The Elysium
supports decoding of BCH-coded data in either error detection or error
correction modes.

The data randomizer specified in the Synchronization and Channel Coding
Sublayer is also optionally supported.

\section{Registers}
\label{sec:sdlpregs}

This section defines the registers in \tabref{sdlpregs}, which apply to the
SDLP Data Link Layer.

\setcounter{regaddr}{192}
\begin{longtable}{Rcr}
		\caption{SDLP Registers}\\
		\label{tab:sdlpregs}\\
		\toprule
		\em Address & \em Name & \em Description\\
		\midrule
		\endhead\EndTableHeader
		\setlabref{reg:tflength} & \labref{TFLength0} & \labref{Transfer Frame
			Length LSB}\\
		\setlabref{reg:tflength} & \labref{TFLength1} & \labref{Transfer Frame
			Length MSB}\\
		\setlabref{reg:tmfec} & \labref{TMFEC} & \labref{Telemetry FEC
			Configuration Bitfields}\\
		\setlabref{reg:tcfec} & \labref{TCFEC} & \labref{Telecommand FEC
			Configuration Bitfields}\\
		\setlabref{reg:sdlpoptions} & \labref{Options} & \labref{General
			Configuration Bitfields}\\
		\setlabref{reg:maxpacketlength} & \labref{MaxPacketLength0} & 
			\labref{Maximum Packet Size LSB}\\
		\setlabref{reg:maxpacketlength} & \labref{MaxPacketLength1} & 
			\labref{Maximum Packet Size MSB}\\
		\setlabref{reg:windowlength} & \labref{WindowLength} & 
			\labref{FARM Sliding Window Length}\\
		\setlabref{reg:maximuminterval} & \labref{MaximumInterval} & 
			\labref{FARM Maximum Reporting Interval}\\
		\setlabref{reg:ids} & \labref{IDs0} & 
			\labref{Spacecraft and Virtual Channel IDs LSB}\\
		\setlabref{reg:ids} & \labref{IDs1} & 
			\labref{Spacecraft and Virtual Channel IDs MSB}\\
		\setlabref{reg:feclvl} & \labref{FECLvl} & 
			\labref{FEC Error Reporting Level}\\
		\setlabref{reg:fecflvl} & \labref{FECFLvl} & 
			\labref{FECF Error Reporting Level}\\
		\setlabref{reg:missedframelvl} & \labref{MissedFrameLvl} & 
			\labref{Missed Frame Error Reporting Level}\\
		\setlabref{reg:lockoutlvl} & \labref{LockoutLvl} & 
			\labref{Lockout Error Reporting Level}\\
		\setlabref{reg:doubleframelvl} & \labref{DoubleFrameLvl} & 
			\labref{Duplicate Frame Error Reporting Level}\\
		\setlabref{reg:invalididlvl} & \labref{InvalidIDLvl} & 
			\labref{Invalid ID Error Reporting Level}\\
		\setlabref{reg:shortframelvl} & \labref{ShortFrameLvl} & 
			\labref{Short Frame Error Reporting Level}\\
		\setlabref{reg:longframelvl} & \labref{LongFrameLvl} & 
			\labref{Long Frame Error Reporting Level}\\
		\setlabref{reg:waitlvl} & \labref{WaitLvl} & 
			\labref{Wait State Error Reporting Level}\\
		\setlabref{reg:sdlperrrpt} & \labref{SDLPErrRpt} & 
			\labref{SDLP Error Reporting Bitfields}\\
		\bottomrule
\end{longtable}
\setcounter{regaddr}{192}

\stoptocentries

\subsection{TFLength[0-1]}

\noindent \textbf{Address:} \elyregaddr[2]

\noindent \textbf{Data Type:} uint16\_t

\noindent \textbf{Description:} The TFLength register contains the length of a
downlink (TM) transfer frame as a 16-bit unsigned integer in bytes.

\noindent \textbf{Diagram:}

\begin{register}{H}{TFLength}{\curreg}
		\label{reg:tflength}
		\regfield{TFLength1}{8}{8}{0000_1000}
		\regfield{TFLength0}{8}{0}{0000_0000}
		\reglabel{2048 Bytes}
\end{register}

\noindent \textbf{Fields:}

\begin{itemize}
		\item TFLength1 - MSB - \curreg[1]
		\item TFLength0 - LSB - \curreg[0]
\end{itemize}

\noindent \textbf{Recommended Value:} 1024 bytes, unless required by FEC.

\noindent \textbf{Notes:} The valid range for this register is from 7 to 2048
bytes. Other restrictions may be imposed by the choice of error correction
coding for the TM link.

\subsection{TMFEC}

\noindent \textbf{Address:} \elyregaddr

\noindent \textbf{Data Type:} Bitfields

\noindent \textbf{Description:} The TMFEC register contains the settings for
Forward Error Correction coding for the TM link.

\noindent \textbf{Diagram:}

\begin{register}{H}{TMFEC}{\curreg}
		\label{reg:tmfec}
		\regfield{FECF}{1}{7}{0}
		\regfield{RS Errors}{1}{6}{0000}
		\regfield{RS Interleave}{3}{3}{0000}
		\regfield{Conv. Rate}{3}{0}{000}
		\reglabel{No Error Control}
\end{register}

\noindent \textbf{Fields:}

\begin{itemize}
		\item FECF - Frame Error Control Field - \curreg.7
		\item RS Errors - Reed-Solomon Error Correction - \curreg.6
		\item RS Interleave - Reed-Solomon Interleaving - \curreg.3
		\item Conv. Rate - Convolutional Coding Rate - \curreg.0
\end{itemize}

\noindent \textbf{Recommended Value:} Rate 1/2 Convolutional coding, without
Frame Error Control Field.

\noindent \textbf{Notes:} When the FECF bit is set, the Frame Error Control
Field is used to detect errors in Transfer Frames on the TM channel. This
setting is independent of other FEC settings.

The RS Errors bit controls the level of error correction provided by the
Reed-Solomon encoding schemes. When the RS Errors bit is set, the RS code used
can correct 16 symbol errors. When the RS Errors bit is not set, the RS code
used can correct 8 symbol errors.

The RS Intervleave bitfield controls the level of interleaving being performed
by the Reed-Solomon encoder. The values of this bitfield are as shown below.

\begin{itemize}
		\item[000 -] RS encoding is entirely disabled.
		\item[001 -] Interleave depth of 1
		\item[010 -] Interleave depth of 2
		\item[011 -] Interleave depth of 3
		\item[100 -] Interleave depth of 4
		\item[101 -] Interleave depth of 5
		\item[110 -] Interleave depth of 8
		\item[111 -] Invalid - clipped to 110
\end{itemize}

The Conv. Rate bitfield specifies the rate used for the (possibly punctured)
convolutional coding applied to TM frames. The values of this bitfield are as
shown below.

\begin{itemize}
		\item[000 -] Convolutional coding is entirely disabled
		\item[001 -] Coding is punctured to rate 7/8 
		\item[010 -] Coding is punctured to rate 5/6
		\item[011 -] Coding is punctured to rate 3/4
		\item[100 -] Coding is punctured to rate 2/3
		\item[101 -] Unpunctured rate 1/2 coding is used
		\item[110 -] Invalid - clipped to 101
		\item[111 -] Invalid - clipped to 101
\end{itemize}

Both the convolutional and Reed-Solomon codes may be used simultaneously
(concatenated coding). In this case, the Reed-Solomon code is the
\textsl{outer} code, while the convolutional code is the \textsl{inner} code

See \href{https://public.ccsds.org/Pubs/131x0b2ec1.pdf}{the TM
Synchronization and Channel Coding standard} for more details of the specific
codes used.

Only rate 1/2 convolutional coding is supported at the time of this writer.
Contact Adamant if you require support for punctured or Reed-Solomon coding.

\subsection{TCFEC}

\noindent \textbf{Address:} \elyregaddr

\noindent \textbf{Data Type:} Bitfields

\noindent \textbf{Description:} The TCFEC register contains the settings for
Forward Error Correction coding for the TC link.

\noindent \textbf{Diagram:}

\begin{register}{H}{TCFEC}{\curreg}
		\label{reg:tcfec}
		\regfield{FECF}{1}{7}{0}
		\regfield{Reserved}{5}{3}{00000}
		\regfield{BCH Mode}{2}{0}{00}
		\reglabel{No Error Control}
\end{register}

\noindent \textbf{Fields:}

\begin{itemize}
		\item FECF - Frame Error Control Field - \curreg.7
		\item Res. - Reserved. These bits are ignored. - \curreg.3
		\item BCH Mode - BCH decoding mode - \curreg.0
\end{itemize}

\noindent \textbf{Recommended Value:} Error correcting BCH coding, without
Frame Error Control Field.

\noindent \textbf{Notes:} When the FECF bit is set, the Frame Error Control
Field is used to detect errors in Transfer Frames on the TC channel. This
setting is independent of other FEC settings.

The reserved bits are ignored - they may be safely set to any value.

The BCH Mode bitfield controls the operating mode of the BCH decoder. The
values of this bitfield are as shown below.

\begin{itemize}
		\item[00 -] BCH encoding is not used, BCH decoder is entirely disabled
		\item[01 -] BCH decoder operates in Triple Error Detection (TED) mode
		\item[10 -] BCH decoder operates in Single Error Correction (SEC) mode
		\item[11 -] Invalid - clipped to 11
\end{itemize}

See \href{https://public.ccsds.org/Pubs/231x0b2c1.pdf}{the TC
Synchronization and Channel Coding standard} for more details of the specific
codes used.

\subsection{Options}

\noindent \textbf{Address:} \elyregaddr

\noindent \textbf{Data Type:} Bitfields

\noindent \textbf{Description:} The Options register contains the settings for
optional components of both the TM and TC links, other than FEC.

\noindent \textbf{Diagram:}

\begin{register}{H}{Options}{\curreg}
		\label{reg:sdlpoptions}
		\regfield{Reserved}{2}{6}{00}
		\regfield{Timestamp Frames}{1}{5}{0}
		\regfield{COP-1/FARM}{1}{4}{0}
		\regfield{Compliant CTLU Start Sequence}{1}{3}{1}
		\regfield{Compliant ASM}{1}{2}{1}
		\regfield{TC Data Whitening}{1}{1}{0}
		\regfield{TM Data Whitening}{1}{0}{0}
		\reglabel{0x0C}
\end{register}

\noindent \textbf{Fields:}

\begin{itemize}
		\item Res. - Reserved. These bits are ignored. - \curreg.6
		\item Timestamp Frames - Optional packet timestamping - \curreg.5
		\item COP-1/FARM - Enable uplink delivery assurance - \curreg.4
		\item Compliant CTLU Start Sequence - Standard uplink sync word -
				\curreg.3
		\item Compliant ASM - Standard downlink sync word - \curreg.2
		\item TC Data Whitening - Data whitening on uplink - \curreg.1
		\item TM Data Whitening - Data whitening on downlink - \curreg.0
\end{itemize}

\noindent \textbf{Recommended Value:} Data whitening on uplink and downlink,
compliant sync words, FARM active, no timestamp.

\noindent \textbf{Notes:} The reserved bits are ignored - they may be safely
set to any value.

When the Timestamp Packets bit is set, all TM frames sent by the Elysium will
use the MC\_FSH service to include a 4-byte timestamp of the Mission Time at
which the frame was sent.

When the COP-1/FARM bit is set, the Frame Acceptance and Reporting Mechanism
(FARM) specified in \href{https://public.ccsds.org/Pubs/232x1b2.pdf}{the COP-1
standard} is used to allow for ARQ functionality on the uplink Virtual Channel.
Note that the Communications Link Control Words (CLCWs) required by COP-1 are
transmitted using the VC\_OCF service on downlink Virtual Channel 0.

When the Compliant CTLU Start Sequence bit is set, the sync word used by the
receiver to detect uplinked frames will be set to the standard value of 0xEB90
specified in \href{https://public.ccsds.org/Pubs/231x0b2c1.pdf}{the TC
Synchronization and Channel Coding standard}, regardless of the setting in
the RXSync register.

When the Compliant ASM bit is set, the sync word added by the transmitter to
allow the ground station to detect downlinked frames will be set to the
value specified in \href{https://public.ccsds.org/Pubs/131x0b2ec1.pdf}{the TM
Synchronization and Channel Coding standard} for the current FEC configuration,
regardless of the setting in the TXSync register.

When the TC Data Whitening bit is set, the data dewhitening procedure specified
in the TC Synchronization and Channel Coding standard will be applied to all
received TC frames.

When the TM Data Whitening bit is set, the data whitening procedure specified
in the TM Synchronization and Channel Coding standard will be applied to all
transmitted TM frames.

See the relevant standards for more details.

\subsection{MaxPacketLength[0-1]}

\noindent \textbf{Address:} \elyregaddr[2]

\noindent \textbf{Data Type:} uint16\_t

\noindent \textbf{Description:} The MaxPacketLength register contains the
maximum length of a packet transferred over the SDLP Data Link Layer as a
16-bit unsigned integer in bytes, minus one.

\noindent \textbf{Diagram:}

\begin{register}{H}{MaxPacketLength}{\curreg}
		\label{reg:maxpacketlength}
		\regfield{MaxPacketLength1}{8}{8}{1111_1111}
		\regfield{MaxPacketLength0}{8}{0}{1111_1111}
		\reglabel{65536 Bytes}
\end{register}

\noindent \textbf{Fields:}

\begin{itemize}
		\item MaxPacketLength1 - MSB - \curreg[1]
		\item MaxPacketLength0 - LSB - \curreg[0]
\end{itemize}

\noindent \textbf{Recommended Value:} Depends on Network Layer protocol - at
least 1500 bytes for IP-based protocols, 8192 bytes for the Space Packet
Protocol.

\noindent \textbf{Notes:} This register contains the maximum length of a packet
\textsl{minus one}. That is, a value of 127 in this register allows a packet of
length 128.

The valid range for this register is from 0 to 65535, or from 1 to 65536 bytes.
Other restrictions may be imposed by the choice of Network Layer.

\subsection{WindowLength}

\noindent \textbf{Address:} \elyregaddr

\noindent \textbf{Data Type:} uint8\_t

\noindent \textbf{Description:} The WindowLength register contains the
length of the sliding window used by the FARM mechanism for ARQ functionality.
It must be an even number.

\noindent \textbf{Diagram:}

\begin{register}{H}{WindowLength}{\curreg}
		\label{reg:windowlength}
		\regfield{WindowLength}{8}{0}{1111_1110}
		\reglabel{254 Frames}
\end{register}

\noindent \textbf{Fields:}

\begin{itemize}
		\item WindowLength - Window length - \curreg[0]
\end{itemize}

\noindent \textbf{Recommended Value:} Depends on mission requirements - 16 is
often appropriate.

\noindent \textbf{Notes:} This is the total window length used by the FARM. The
window is divided into a positive and a negative region, with sequence numbers
falling outside the window being in the Lockout region. Because of this, the
length of the window must be an even number of bytes.

See \href{https://public.ccsds.org/Pubs/232x1b2.pdf}{the COP-1 standard} for
more details.

\subsection{MaximumInterval}

\noindent \textbf{Address:} \elyregaddr

\noindent \textbf{Data Type:} uint8\_t (s)

\noindent \textbf{Description:} The MaximumInterval register contains the
maximum length of time which the FARM may go without reporting back to the
ground station through the Communications Link Control Word (CLCW).

\noindent \textbf{Diagram:}

\begin{register}{H}{MaximumInterval}{\curreg}
		\label{reg:maximuminterval}
		\regfield{MaximumInterval}{8}{0}{1111_1111}
		\reglabel{255 Seconds}
\end{register}

\noindent \textbf{Fields:}

\begin{itemize}
		\item MaximumInterval - Maximum FARM reporting interval - \curreg[0]
\end{itemize}

\noindent \textbf{Recommended Value:} Depends on mission requirements - 16 is
often appropriate.

\noindent \textbf{Notes:} If an uplink frame has been received and no downlink
frame has been sent for the duration specified in this register, an Only Idle
Data (OID) frame will be transmitted containing the required CLCW.

See the TC and COP-1 standards for more details.

\subsection{IDs[0-1]}

\noindent \textbf{Address:} \elyregaddr[2]

\noindent \textbf{Data Type:} Bitfields

\noindent \textbf{Description:} The IDs register contains the
Spacecraft ID and the Virtual Channel IDs used by the SDLP Data Link Layer.

\noindent \textbf{Diagram:}

\begin{register}{H}{IDs}{\curreg}
		\label{reg:ids}
		\regfield{VCID1}{3}{13}{011}
		\regfield{VCID0}{3}{10}{111}
		\regfield{SCID Bits 8-9}{2}{8}{00}
		\regfield{SCID Bits 0-7}{8}{0}{0010_1111}
		\reglabel{SCID 0x2F VCID0 0x07 VCID1 0x03}
\end{register}

\noindent \textbf{Fields:}

\begin{itemize}
		\item VCID1 - 3-bit Virtual Channel ID for VC1 - \curreg[1].5
		\item VCID0 - 3-bit Virtual Channel ID for VC0 - \curreg[1].2
		\item SCID - 10-bit Spacecraft ID - \curreg[0]
\end{itemize}

\noindent \textbf{Recommended Value:} N/A.

\noindent \textbf{Notes:} The same Spacecraft ID is used for both uplink and
downlink.

The downlink Data Link Layer uses two Virtual Channels, identified by two 3-bit
VCIDs, VCID0 and VCID1. The uplink Data Link Layer uses only one Virtual
Channel, identified by one 6-bit VCID formed by concatenating VCID0 and VCID1
so that the most significant bit of VCID1 becomes the most significant bit of
the new VCID and the least significant bit of VCID0 becomes the least
significant bit of the new VCID.

\subsection{FECLvl}

\noindent \textbf{Address:} \elyregaddr

\noindent \textbf{Data Type:} Priority Enumeration

\noindent \textbf{Description:} The FECLvl register controls the priority
level of FEC error correction or detection events, if enabled by the FECRpt
bit of the \hyperref[reg:sdlperrrpt]{SDLPErrRpt register}.

\noindent \textbf{Diagram:}

\begin{register}{H}{FECLvl}{\curreg}
		\label{reg:feclvl}
		\regfield{FECLvl}{8}{0}{00000100}
		\reglabel{WARNING}
\end{register}

\noindent \textbf{Fields:}

\begin{itemize}
		\item FECLvl - Priority level of FEC error correction or detection
				errors - \curreg
\end{itemize}

\noindent \textbf{Recommended Value:} WARNING

\noindent \textbf{Notes:} The acceptable values for this register are the valid
values of the Priority Enumeration data type, a one-hot encoding using bits 0
through 4.

\subsection{FECFLvl}

\noindent \textbf{Address:} \elyregaddr

\noindent \textbf{Data Type:} Priority Enumeration

\noindent \textbf{Description:} The FECFLvl register controls the priority
level of FECF error detection events, if enabled by the FECFRpt bit of the
\hyperref[reg:sdlperrrpt]{SDLPErrRpt register}.

\noindent \textbf{Diagram:}

\begin{register}{H}{FECFLvl}{\curreg}
		\label{reg:fecflvl}
		\regfield{FECFLvl}{8}{0}{00001000}
		\reglabel{ERROR}
\end{register}

\noindent \textbf{Fields:}

\begin{itemize}
		\item FECFLvl - Priority level of FECF error detection errors - \curreg
\end{itemize}

\noindent \textbf{Recommended Value:} ERROR

\noindent \textbf{Notes:} The acceptable values for this register are the valid
values of the Priority Enumeration data type, a one-hot encoding using bits 0
through 4.

\subsection{MissedFrameLvl}

\noindent \textbf{Address:} \elyregaddr

\noindent \textbf{Data Type:} Priority Enumeration

\noindent \textbf{Description:} The MissedFrameLvl register controls the
priority level of FARM missed frame detection events, if enabled by the
MissedFrameRpt bit of the \hyperref[reg:sdlperrrpt]{SDLPErrRpt register}.

\noindent \textbf{Diagram:}

\begin{register}{H}{MissedFrameLvl}{\curreg}
		\label{reg:missedframelvl}
		\regfield{MissedFrameLvl}{8}{0}{00001000}
		\reglabel{ERROR}
\end{register}

\noindent \textbf{Fields:}

\begin{itemize}
		\item MissedFrameLvl - Priority level of FARM missed frame detection 
				errors - \curreg
\end{itemize}

\noindent \textbf{Recommended Value:} ERROR

\noindent \textbf{Notes:} The acceptable values for this register are the valid
values of the Priority Enumeration data type, a one-hot encoding using bits 0
through 4.

\subsection{LockoutLvl}

\noindent \textbf{Address:} \elyregaddr

\noindent \textbf{Data Type:} Priority Enumeration

\noindent \textbf{Description:} The LockoutLvl register controls the priority
level of FARM lockout events, if enabled by the LockoutRpt bit of the
\hyperref[reg:sdlperrrpt]{SDLPErrRpt register}.

\noindent \textbf{Diagram:}

\begin{register}{H}{LockoutLvl}{\curreg}
		\label{reg:lockoutlvl}
		\regfield{LockoutLvl}{8}{0}{00001000}
		\reglabel{ERROR}
\end{register}

\noindent \textbf{Fields:}

\begin{itemize}
		\item LockoutLvl - Priority level of FARM lockout errors - \curreg
\end{itemize}

\noindent \textbf{Recommended Value:} ERROR

\noindent \textbf{Notes:} The acceptable values for this register are the valid
values of the Priority Enumeration data type, a one-hot encoding using bits 0
through 4.

\subsection{DoubleFrameLvl}

\noindent \textbf{Address:} \elyregaddr

\noindent \textbf{Data Type:} Priority Enumeration

\noindent \textbf{Description:} The DoubleFrameLvl register controls the
priority level of FARM multiple reception events, if enabled by the
DoubleFrameLvl bit of the \hyperref[reg:sdlperrrpt]{SDLPErrRpt register}.

\noindent \textbf{Diagram:}

\begin{register}{H}{DoubleFrameLvl}{\curreg}
		\label{reg:doubleframelvl}
		\regfield{DoubleFrameLvl}{8}{0}{00000010}
		\reglabel{INFO}
\end{register}

\noindent \textbf{Fields:}

\begin{itemize}
		\item DoubleFrameLvl - Priority level of duplicate frame reception 
				errors - \curreg
\end{itemize}

\noindent \textbf{Recommended Value:} INFO

\noindent \textbf{Notes:} The acceptable values for this register are the valid
values of the Priority Enumeration data type, a one-hot encoding using bits 0
through 4.

\subsection{InvalidIDLvl}

\noindent \textbf{Address:} \elyregaddr

\noindent \textbf{Data Type:} Priority Enumeration

\noindent \textbf{Description:} The InvalidIDLvl register controls the
priority level of invalid Transfer Frame ID reception events, if enabled by the
InvalidIDRpt bit of the \hyperref[reg:sdlperrrpt]{SDLPErrRpt register}.

\noindent \textbf{Diagram:}

\begin{register}{H}{InvalidIDLvl}{\curreg}
		\label{reg:invalididlvl}
		\regfield{InvalidIDLvl}{8}{0}{00001000}
		\reglabel{ERROR}
\end{register}

\noindent \textbf{Fields:}

\begin{itemize}
		\item InvalidIDLvl - Priority level of invalid Transfer Frame ID 
				reception errors - \curreg
\end{itemize}

\noindent \textbf{Recommended Value:} ERROR

\noindent \textbf{Notes:} The acceptable values for this register are the valid
values of the Priority Enumeration data type, a one-hot encoding using bits 0
through 4.

\subsection{ShortFrameLvl}

\noindent \textbf{Address:} \elyregaddr

\noindent \textbf{Data Type:} Priority Enumeration

\noindent \textbf{Description:} The ShortFrameLvl register controls the
priority level of invalid Transfer Frame length events where fewer bytes are
received than indicated by the Transfer Frame Length field, if enabled by the
LengthRpt bit of the \hyperref[reg:sdlperrrpt]{SDLPErrRpt register}.

\noindent \textbf{Diagram:}

\begin{register}{H}{ShortFrameLvl}{\curreg}
		\label{reg:shortframelvl}
		\regfield{ShortFrameLvl}{8}{0}{00001000}
		\reglabel{ERROR}
\end{register}

\noindent \textbf{Fields:}

\begin{itemize}
		\item ShortFrameLvl - Priority level of short Transfer Frame reception
				errors - \curreg
\end{itemize}

\noindent \textbf{Recommended Value:} ERROR

\noindent \textbf{Notes:} The acceptable values for this register are the valid
values of the Priority Enumeration data type, a one-hot encoding using bits 0
through 4.

\subsection{LongFrameLvl}

\noindent \textbf{Address:} \elyregaddr

\noindent \textbf{Data Type:} Priority Enumeration

\noindent \textbf{Description:} The LongFrameLvl register controls the
priority level of invalid Transfer Frame length events where more bytes are
received than indicated by the Transfer Frame Length field, if enabled by the
LengthRpt bit of the \hyperref[reg:sdlperrrpt]{SDLPErrRpt register}.

\noindent \textbf{Diagram:}

\begin{register}{H}{LongFrameLvl}{\curreg}
		\label{reg:longframelvl}
		\regfield{LongFrameLvl}{8}{0}{00000100}
		\reglabel{WARNING}
\end{register}

\noindent \textbf{Fields:}

\begin{itemize}
		\item LongFrameLvl - Priority level of long Transfer Frame reception
				errors - \curreg
\end{itemize}

\noindent \textbf{Recommended Value:} WARNING

\noindent \textbf{Notes:} The acceptable values for this register are the valid
values of the Priority Enumeration data type, a one-hot encoding using bits 0
through 4.

\subsection{WaitLvl}

\noindent \textbf{Address:} \elyregaddr

\noindent \textbf{Data Type:} Priority Enumeration

\noindent \textbf{Description:} The WaitLvl register controls the priority
level of FARM insufficient buffer events, if enabled by the WaitRpt bit of the
\hyperref[reg:sdlperrrpt]{SDLPErrRpt register}.

\noindent \textbf{Diagram:}

\begin{register}{H}{WaitLvl}{\curreg}
		\label{reg:waitlvl}
		\regfield{WaitLvl}{8}{0}{00001000}
		\reglabel{ERROR}
\end{register}

\noindent \textbf{Fields:}

\begin{itemize}
		\item WaitLvl - Priority level of FARM insufficient buffer errors - 
				\curreg
\end{itemize}

\noindent \textbf{Recommended Value:} ERROR

\noindent \textbf{Notes:} The acceptable values for this register are the valid
values of the Priority Enumeration data type, a one-hot encoding using bits 0
through 4.

\subsection{SDLPErrRpt}

\noindent \textbf{Address:} \elyregaddr

\noindent \textbf{Data Type:} Bitfields

\noindent \textbf{Description:} The SDLPErrRpt register contains a number of
bitfields controlling the reporting of errors within the SDLP Data Link Layer.

\noindent \textbf{Diagram:}

\begin{register}{H}{SDLPErrRpt}{\curreg}
		\label{reg:sdlperrrpt}
		\regfield{FECRpt}{1}{7}{1}
		\regfield{FECFRpt}{1}{6}{1}
		\regfield{MissedFrameRpt}{1}{5}{1}
		\regfield{LockoutRpt}{1}{4}{1}
		\regfield{DoubleFrameRpt}{1}{3}{1}
		\regfield{InvalidIDRpt}{1}{2}{1}
		\regfield{LengthRpt}{1}{1}{1}
		\regfield{WaitRpt}{1}{0}{1}
		\reglabel{All Errors Reported}
\end{register}

\noindent \textbf{Fields:}

\begin{itemize}
		\item FECRpt - Enables reporting of FEC error correction or detection 
				- \curreg[0].7
		\item FECFRpt - Enables reporting of FECF error detection - 
				\curreg[0].6
		\item MissedFrameRpt - Enables reporting of missed frame errors - 
				\curreg[0].5
		\item LockoutRpt - Enables reporting of FARM lockout errors - 
				\curreg[0].4
		\item DoubleFrameRpt - Enables reporting of repeat frame reception 
				errors - \curreg[0].3
		\item InvalidIDRpt - Enables reporting of invalid Transfer Frame ID
				errors - \curreg[0].2
		\item LengthRpt - Enables reporting of incorrect Transfer Frame Length
				errors - \curreg[0].1
		\item WaitRpt - Enables reporting of FARM insufficient buffer errors - 
				\curreg[0].0
\end{itemize}

\noindent \textbf{Recommended Value:} If the spacecraft contains a flight
computer which is capable of taking action to correct errors, in general all
errors should be reported.

\noindent \textbf{Notes:} When the FECRpt bit is set, anytime the FEC decoder
decoding the binary BCH coding on the uplink channel corrects or detects an
error, the decoder will report an error with the priority level defined in the
\hyperref[reg:feclvl]{FECLvl register}.

When the FECFRpt bit is set, anytime the Frame Error Control Field check on the
uplink channel detects an error, an error will be reported with the priority
level defined in the \hyperref[reg:fecflvl]{FECFLvl register}.

When the MissedFrameRpt bit is set, anytime the FARM on the uplink channel
detects a missed frame, an error will be reported with the priority
level defined in the \hyperref[reg:missedframelvl]{MissedFrameLvl register}.

When the LockoutRpt bit is set, anytime the FARM on the uplink channel receives
a Transfer Frame which sends it into the Lockout state, an error will be
reported with the priority level defined in the
\hyperref[reg:lockoutlvl]{LockoutLvl register}.

When the DoubleFrameRpt bit is set, anytime the FARM on the uplink channel
detects a repeat frame, an error will be reported with the priority
level defined in the \hyperref[reg:doubleframelvl]{DoubleFrameLvl register}.

When the InvalidIDRpt bit is set, anytime a Transfer Frame is received on the
uplink channel which has an invalid SCID or VCID, an error will be reported
with the priority level defined in the \hyperref[reg:invalididlvl]{InvalidIDLvl
register}.

When the LengthRpt bit is set, anytime a Transfer Frame is received on the
uplink channel which contains a number of bytes different than the number
specified in the Transfer Frame Length field, an error will be reported.
The error will have the priority level defined in the
\hyperref[reg:shortframelvl]{ShortFrameLvl register} if fewer bytes are
received than expected, or the priority level defined in the
\hyperref[reg:longframelvl]{LongFrameLvl register} if more bytes are received
than expected.

When the WaitRpt bit is set, when the FARM on the uplink channel receives a
Transfer Frame for which no buffer is available, sending the FARM state machine
into the Wait state, an error will be reported with the priority level defined
in the \hyperref[reg:waitlvl]{WaitLvl register}.

\starttocentries

\section{Channels}
\label{sec:sdlpchan}

\setcounter{idval}{112}
\begin{longtable}{Wcr}
		\caption{Channels}\\
		\label{tab:channels}\\
		\toprule
		\em ID & \em Name  & \em Data Type\\
		\midrule
		\endhead\EndTableHeader
		\setlabref{chan:sdlpfec} & \labref{FEC Corrections/Detections} & 
			\labref{uint8\_t}\\
		\setlabref{chan:sdlpcrc} & \labref{FECF Detections} & 
			\labref{uint8\_t}\\
		\setlabref{chan:sdlpvr} & \labref{Receiver Frame Sequence Number V(R)} & 
			\labref{uint8\_t}\\
		\setlabref{chan:sdlpmissed} & \labref{Missed Frames} & 
			\labref{uint8\_t}\\
		\setlabref{chan:sdlpdouble} & \labref{Double Receptions} & 
			\labref{uint8\_t}\\
		\setlabref{chan:sdlpas} & \labref{Type-A Frames Received} & 
			\labref{uint8\_t}\\
		\setlabref{chan:sdlpbs} & \labref{Type-B Frames Received} & 
			\labref{uint8\_t}\\
		\bottomrule
\end{longtable}
\setcounter{idval}{112}

\stoptocentries

\subsection{FEC Corrections/Detections}
\label{chan:sdlpfec}

\noindent \textbf{Channel ID:} \elyid 

\noindent \textbf{Data Type:} uint8\_t

\noindent \textbf{Description:} The FEC Corrections/Detections Channel reports
the number of FEC corrections or detections (depending on the value of the
BCH Mode field of the \hyperref[reg:tcfec]{TCFEC register}). 

\noindent \textbf{Format:}
\newline
\newline
\begin{center}
\begin{bytefield}[endianness=big,bitwidth=2.25em]{8}
		\bitheader{0-7}\\
		\begin{rightwordgroup}{ID}
				\bitbox{8}{\currid}
		\end{rightwordgroup}\\
		\begin{rightwordgroup}{FEC Corrections}
				\bitbox{8}{Value}
		\end{rightwordgroup}
\end{bytefield}
\end{center}

\noindent \textbf{Notes:} This counter increases monotonically until it rolls
over from 255 to 0.

\subsection{FECF Detections}
\label{chan:sdlpcrc}

\noindent \textbf{Channel ID:} \elyid 

\noindent \textbf{Data Type:} uint8\_t

\noindent \textbf{Description:} The FECF Detections Channel reports the number
of errors detected by the CRC value stored in the Frame Error Control Field of
uplink frames, if enabled by the FECF field of the \hyperref[reg:tcfec]{TCFEC
register}.

\noindent \textbf{Format:}
\newline
\newline
\begin{center}
\begin{bytefield}[endianness=big,bitwidth=2.25em]{8}
		\bitheader{0-7}\\
		\begin{rightwordgroup}{ID}
				\bitbox{8}{\currid}
		\end{rightwordgroup}\\
		\begin{rightwordgroup}{FECF Detections}
				\bitbox{8}{Value}
		\end{rightwordgroup}
\end{bytefield}
		\end{center}

\noindent \textbf{Notes:} This counter increases monotonically until it rolls
over from 255 to 0.

\subsection{Receiver Frame Sequence Number V(R)}
\label{chan:sdlpvr}

\noindent \textbf{Channel ID:} \elyid 

\noindent \textbf{Data Type:} uint8\_t

\noindent \textbf{Description:} The Receiver Frame Sequence Number Channel
contains the Receiver Frame Sequence Number, called V(R) in the TC and COP-1
standards. This is the value of Frame Sequence Number, or N(S), expected to be
seen in the next Type-AD Transfer Frame on the uplink Virtual Channel.

\noindent \textbf{Format:}
\newline
\newline
\begin{center}
\begin{bytefield}[endianness=big,bitwidth=2.25em]{8}
		\bitheader{0-7}\\
		\begin{rightwordgroup}{ID}
				\bitbox{8}{\currid}
		\end{rightwordgroup}\\
		\begin{rightwordgroup}{V(R)}
				\bitbox{8}{Value}
		\end{rightwordgroup}
\end{bytefield}
				\end{center}

\noindent \textbf{Notes:} This counter increases monotonically as Transfer
Framse are accepted until it rolls over from 255 to 0 or is modified by a "Set
V(R)" command as defined in the COP-1 standard.

\subsection{Missed Frames}
\label{chan:sdlpmissed}

\noindent \textbf{Channel ID:} \elyid 

\noindent \textbf{Data Type:} uint8\_t

\noindent \textbf{Description:} The Missed Frames Channel contains the number
of frames that have been missed on the uplink channel as detected by the FARM.

\noindent \textbf{Format:}
\newline
\newline
\begin{center}
\begin{bytefield}[endianness=big,bitwidth=2.25em]{8}
		\bitheader{0-7}\\
		\begin{rightwordgroup}{ID}
				\bitbox{8}{\currid}
		\end{rightwordgroup}\\
		\begin{rightwordgroup}{Missed Frames}
				\bitbox{8}{Value}
		\end{rightwordgroup}
\end{bytefield}
						\end{center}

\noindent \textbf{Notes:} When a Transfer Frame is received which has a
sequence number greater than the current value of V(R), this counter is
increased by the difference between the received sequence number N(S) and the
current value of V(R).

This counter increases monotonically until it rolls over from 255 to 0.

\subsection{Double Receptions}
\label{chan:sdlpdouble}

\noindent \textbf{Channel ID:} \elyid 

\noindent \textbf{Data Type:} uint8\_t

\noindent \textbf{Description:} The Double Receptions Channel contains the
number of frames that have been received twice on the uplink channel as
detected by the FARM.

\noindent \textbf{Format:}
\newline
\newline
\begin{center}
\begin{bytefield}[endianness=big,bitwidth=2.25em]{8}
		\bitheader{0-7}\\
		\begin{rightwordgroup}{ID}
				\bitbox{8}{\currid}
		\end{rightwordgroup}\\
		\begin{rightwordgroup}{Double Receptions}
				\bitbox{8}{Value}
		\end{rightwordgroup}
\end{bytefield}
								\end{center}

\noindent \textbf{Notes:} When a Transfer Frame is received which has a
sequence number less than the current value of V(R), this counter is
increased by one.

This counter increases monotonically until it rolls over from 255 to 0.

\subsection{Type-A Frames Received}
\label{chan:sdlpas}

\noindent \textbf{Channel ID:} \elyid 

\noindent \textbf{Data Type:} uint8\_t

\noindent \textbf{Description:} The Type A Frames Received Channel contains the
number of Type-A frames that have been received on the uplink channel.

\noindent \textbf{Format:}
\newline
\newline
\begin{center}
\begin{bytefield}[endianness=big,bitwidth=2.25em]{8}
		\bitheader{0-7}\\
		\begin{rightwordgroup}{ID}
				\bitbox{8}{\currid}
		\end{rightwordgroup}\\
		\begin{rightwordgroup}{Type-A Frames}
				\bitbox{8}{Value}
		\end{rightwordgroup}
\end{bytefield}
										\end{center}

\noindent \textbf{Notes:} This counter is incremented only when a Type-A frame
is fully accepted (i.e., has passed all frame validation checks).

This counter increases monotonically until it rolls over from 255 to 0.

\subsection{Type-B Frames Received}
\label{chan:sdlpbs}

\noindent \textbf{Channel ID:} \elyid 

\noindent \textbf{Data Type:} uint8\_t

\noindent \textbf{Description:} The Type B Frames Received Channel contains the
number of Type-B frames that have been received on the uplink channel.

\noindent \textbf{Format:}
\newline
\newline
\begin{center}
\begin{bytefield}[endianness=big,bitwidth=2.25em]{8}
		\bitheader{0-7}\\
		\begin{rightwordgroup}{ID}
				\bitbox{8}{\currid}
		\end{rightwordgroup}\\
		\begin{rightwordgroup}{Type-B Frames}
				\bitbox{8}{Value}
		\end{rightwordgroup}
\end{bytefield}
												\end{center}

\noindent \textbf{Notes:} This counter is incremented only when a Type-B frame
is fully accepted (i.e., has passed the Frame Error Control Field check, if
enabled).

This counter increases monotonically until it rolls over from 255 to 0.

\starttocentries

\section{Errors}
\label{sec:sdlperrs}

\setcounter{idval}{176}
\begin{longtable}{Wr}
		\caption{Errors}\\
		\label{tab:errs}\\
		\toprule
		\em ID & \em Error\\
		\midrule
		\endhead\EndTableHeader
		\setlabref{err:sdlpfec} & \labref{FEC Correct/Detect}\\
		\setlabref{err:sdlpcrc} & \labref{FECF Detect}\\
		\setlabref{err:sdlpmissed} & \labref{Missed Frame}\\
		\setlabref{err:sdlplockout} & \labref{Lockout}\\
		\setlabref{err:sdlpdouble} & \labref{Double Reception}\\
		\setlabref{err:sdlpids} & \labref{Invalid Frame ID}\\
		\setlabref{err:sdlpshort} & \labref{Frame Too Short}\\
		\setlabref{err:sdlplong} & \labref{Frame Too Long}\\
		\setlabref{err:sdlpwait} & \labref{Wait}\\
		\bottomrule
\end{longtable}
\setcounter{idval}{176}

\stoptocentries

\subsection{FEC Correct/Detect}
\label{err:sdlpfec}

\noindent \textbf{Error ID:} \elyid 

\noindent \textbf{Description:} The FEC Correct/Detect error indicates that the
Forward Error Correction decoder on the uplink channel has corrected or
detected an error (depending on the value of the BCH Mode field of the
\hyperref[reg:tcfec]{TCFEC register}). 

\noindent \textbf{Fault Response?} Transfer Frame discarded.

\noindent \textbf{Recommended Priority:} WARNING for correction, ERROR for
detection.

\noindent \textbf{Priority Register:} \hyperref[reg:feclvl]{FECLvl}

\subsection{FECF Detect}
\label{err:sdlpcrc}

\noindent \textbf{Error ID:} \elyid 

\noindent \textbf{Description:} The FECF Detect error indicates that the
Frame Error Control Field check on the uplink channel has 
detected an error.

\noindent \textbf{Fault Response?} Transfer Frame discarded.

\noindent \textbf{Recommended Priority:} ERROR 

\noindent \textbf{Priority Register:} \hyperref[reg:fecflvl]{FECFLvl}

\subsection{Missed Frame}
\label{err:sdlpmissed}

\noindent \textbf{Error ID:} \elyid 

\noindent \textbf{Description:} The Missed Frame error indicates that the FARM
on the uplink channel has detected one or more missing frames.

\noindent \textbf{Fault Response?} None.

\noindent \textbf{Recommended Priority:} ERROR

\noindent \textbf{Priority Register:}
\hyperref[reg:missedframelvl]{MissedFrameLvl}

\subsection{Lockout}
\label{err:sdlplockout}

\noindent \textbf{Error ID:} \elyid 

\noindent \textbf{Description:} The Lockout error indicates that the FARM
on the uplink channel has received a Transfer Frame with a sequence number
outside of the sliding window and gone in the Lockout mode.

\noindent \textbf{Fault Response?} Transfer Frame discarded. No further
Transfer Frames are accepted until the Unlock command described in the COP-1
standard is received.

\noindent \textbf{Recommended Priority:} ERROR

\noindent \textbf{Priority Register:} \hyperref[reg:lockoutlvl]{LockoutLvl}

\subsection{Double Reception}
\label{err:sdlpdouble}

\noindent \textbf{Error ID:} \elyid 

\noindent \textbf{Description:} The Double Reception error indicates that the
FARM on the uplink channel has received a Transfer Frame with a sequence number
less than the expected sequence number - that is, a Transfer Frame which has
already been received.

\noindent \textbf{Fault Response?} Transfer Frame discarded.

\noindent \textbf{Recommended Priority:} INFO

\noindent \textbf{Priority Register:}
\hyperref[reg:doubleframelvl]{DoubleFrameLvl}

\subsection{Invalid Frame ID}
\label{err:sdlpids}

\noindent \textbf{Error ID:} \elyid 

\noindent \textbf{Description:} The Invalid Frame ID error indicates that the 
radio has received a Transfer Frame with an invalid Spacecraft ID (SCID),
Master Channel ID (MCID), or Virtual Channel ID (VCID).

\noindent \textbf{Fault Response?} Transfer Frame discarded.

\noindent \textbf{Recommended Priority:} ERROR

\noindent \textbf{Priority Register:} \hyperref[reg:invalididlvl]{InvalidIDLvl}

\subsection{Frame Too Short}
\label{err:sdlpshort}

\noindent \textbf{Error ID:} \elyid 

\noindent \textbf{Description:} The Frame Too Short error indicates that the 
radio has received a Transfer Frame whose Length field contains a value larger
than the number of received bytes.

\noindent \textbf{Fault Response?} Transfer Frame discarded.

\noindent \textbf{Recommended Priority:} ERROR

\noindent \textbf{Priority Register:}
\hyperref[reg:shortframelvl]{ShortFrameLvl}

\subsection{Frame Too Long}
\label{err:sdlplong}

\noindent \textbf{Error ID:} \elyid 

\noindent \textbf{Description:} The Frame Too Long error indicates that the 
radio has received a Transfer Frame whose Length field contains a value smaller
than the number of received bytes.

\noindent \textbf{Fault Response?} Excess bytes discarded.

\noindent \textbf{Recommended Priority:} WARNING

\noindent \textbf{Priority Register:} \hyperref[reg:longframelvl]{LongFrameLvl}

\subsection{Wait}
\label{err:sdlpwait}

\noindent \textbf{Error ID:} \elyid 

\noindent \textbf{Description:} The Wait error indicates that the 
radio has received a valid Transfer Frame, but no buffer space is available to
store the Transfer Frame. If the FARM is active, the FARM state machine
transitions to the Wait state.

\noindent \textbf{Fault Response?} Transfer Frame discarded

\noindent \textbf{Recommended Priority:} ERROR

\noindent \textbf{Priority Register:} \hyperref[reg:waitlvl]{WaitLvl}

\starttocentries

\section{Events}
\label{sec:sdlpevts}

\setcounter{idval}{240}
\begin{longtable}{Wr}
		\caption{Events}\\
		\label{tab:events}\\
		\toprule
		\em ID & \em Event\\
		\midrule
		\endhead\EndTableHeader
		\setlabref{evt:sdlpvr} & \labref{Receiver Sequence Number V(R) Changed}\\
		\setlabref{evt:sdlpunlock} & \labref{Unlock}\\
		\setlabref{evt:sdlpunwait} & \labref{Buffer Release}\\
\end{longtable}
\setcounter{idval}{240}

\stoptocentries

\subsection{Receiver Sequence Number V(R) Changed}
\label{evt:sdlpvr}

\noindent \textbf{Event ID:} \elyid 

\noindent \textbf{Description:} The Receiver Sequence Number V(R) Changed Event
indicates that the current value of the Receiver Sequence Number - V(R) as
defined in the COP-1 standard -  has changed in a discontinuous way due to a
"Set V(R)" command as defined in the COP-1 standard.

\noindent \textbf{Notes:} The new value of V(R) can be retrieved from the
\hyperref[chan:sdlpvr]{Receiver Frame Seuqence Number V(R)} channel.

\subsection{Unlock}
\label{evt:sdlpunlock}

\noindent \textbf{Event ID:} \elyid 

\noindent \textbf{Description:} The Unlock Event indicates that the FARM state
machine has received the "Unlock" command as defined in the COP-1 standard and
exited the Lock state, if required.

\noindent \textbf{Notes:} This event is reported whenever the Unlock command is
received, regardless of whether the FARM state machine is in the Lock state.

\subsection{Buffer Release}
\label{evt:sdlpunwait}

\noindent \textbf{Event ID:} \elyid 

\noindent \textbf{Description:} The Buffer Release Event corresponds to the
Buffer Release Signal defined in the COP-1 standard and indicates that buffer
space is now available for new Transfer Frames to arrive. The FARM state
machine has also exited the Wait state.

\noindent \textbf{Notes:} This event is reported only when the FARM state
machine is active.

\starttocentries

\appendix

\section{Revision History}

\begin{enumerate}
		\item Initial release
\end{enumerate}

\end{document}
