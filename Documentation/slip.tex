\documentclass{hitec}

\author{Andrew Wygle}
\title{Elysium Radio Networking Layer - Serial Line Internet Protocol}
\company{Adamant Aerospace}
\confidential{\textbf{Unlimited Distribution}}

\usepackage{graphicx}
\usepackage{booktabs}
\usepackage{pdfpages}
\usepackage{gensymb}
\usepackage{siunitx}
\usepackage{bytefield}
\usepackage[hyperref=true]{register}
\usepackage{rotating}
\usepackage{titlesec}
\usepackage{subcaption}
\usepackage{fmtcount}
\usepackage{parskip}
\usepackage{caption}
\usepackage[autostyle, english = american]{csquotes}
\usepackage{enumitem}
\usepackage{longtable}
\usepackage{array}
\usepackage[colorlinks=true, linktoc=all]{hyperref}
\usepackage{etoolbox}

\MakeOuterQuote{"}

\settextfraction{1}

\newcommand{\chref}[1]{\hyperref[ch:#1]{Section~\ref{ch:#1}}}
\newcommand{\figref}[1]{\hyperref[fig:#1]{Figure~\ref{fig:#1}}}
\newcommand{\tabref}[1]{\hyperref[tab:#1]{Table~\ref{tab:#1}}}
\makeatletter
\newcommand*{\textoverline}[1]{$\overline{\hbox{#1}}\m@th$}
\makeatother

\newcommand{\bitlabel}[2]{%
	\bitbox[]{#1}{%
		\raisebox{0pt}[4ex][0pt]{%
			\turnbox{45}{\fontsize{7}{7}\selectfont#2}%
		}%
	}%
}

\titleclass{\subsubsubsection}{straight}[\subsection]

\newcounter{subsubsubsection}[subsubsection]
\renewcommand\thesubsubsubsection{\thesubsubsection.\arabic{subsubsubsection}}
\renewcommand\theparagraph{\thesubsubsubsection.\arabic{paragraph}} % optional; useful if paragraphs are to be numbered

\titleformat{\subsubsubsection}
  {\normalfont\normalsize\bfseries}{\thesubsubsubsection}{1em}{}
\titlespacing*{\subsubsubsection}
{0pt}{3.25ex plus 1ex minus .2ex}{1.5ex plus .2ex}

\makeatletter
\renewcommand\paragraph{\@startsection{paragraph}{5}{\z@}%
  {3.25ex \@plus1ex \@minus.2ex}%
  {-1em}%
  {\normalfont\normalsize\bfseries}}
\renewcommand\subparagraph{\@startsection{subparagraph}{6}{\parindent}%
  {3.25ex \@plus1ex \@minus .2ex}%
  {-1em}%
  {\normalfont\normalsize\bfseries}}
\def\toclevel@subsubsubsection{4}
\def\toclevel@paragraph{5}
\def\toclevel@paragraph{6}
\def\l@subsubsubsection{\@dottedtocline{4}{7em}{4em}}
\def\l@paragraph{\@dottedtocline{5}{10em}{5em}}
\def\l@subparagraph{\@dottedtocline{6}{14em}{6em}}
\makeatother

\setcounter{secnumdepth}{4}
\setcounter{tocdepth}{4}

\titleformat{\subsubsection}
	{\normalfont\fontsize{12}{15}\bfseries}{\thesubsubsection}{1em}{}


\newcounter{regaddr}
\newcounter{regsize}
\newcommand*{\elyregaddr}[1][1]{0x\padzeroes[2]\Hexadecimal{regaddr}\addtocounter{regaddr}{#1}\setcounter{regsize}{#1}}

\newcommand*{\curreg}[1][0]{%
		\addtocounter{regaddr}{\the\numexpr#1-\value{regsize}\relax}%
		0x\padzeroes[2]\Hexadecimal{regaddr}%
		\addtocounter{regaddr}{\the\numexpr\value{regsize}-#1\relax}%
}

\newcounter{idval}
\newcommand*{\elyid}[0]{0x\padzeroes[2]\Hexadecimal{idval}\addtocounter{idval}{1}}

\newcommand*{\currid}[0]{%
		\addtocounter{idval}{-1}%
		0x\padzeroes[2]\Hexadecimal{idval}%
		\addtocounter{idval}{1}%
}

\newtoggle{inTableHeader}
\toggletrue{inTableHeader}
\newcommand*{\StartTableHeader}{\global\toggletrue{inTableHeader}}
\newcommand*{\EndTableHeader}{\global\togglefalse{inTableHeader}}
\let\OldLongtable\longtable%
\let\OldEndLongtable\endlongtable%
\renewenvironment{longtable}{\StartTableHeader\OldLongtable}{\OldEndLongtable\StartTableHeader}%

\newcommand*{\setlabref}[1]{\gdef\currlabref{#1}}
\newcommand*{\labref}[1]{\iftoggle{inTableHeader}{}{\hyperref[\currlabref]{#1}}}
\newcolumntype{W}{>{\global\let\currlabref\relax}l<{\labref{\elyid}}}
\newcolumntype{R}{>{\global\let\currlabref\relax}l<{\labref{\elyregaddr}}}

\let\oldaddcontentsline\addcontentsline
\newcommand{\stoptocentries}{\renewcommand{\addcontentsline}[3]{}}
\newcommand{\starttocentries}{\let\addcontentsline\oldaddcontentsline}

\begin{document}

% NOTE page numbering is controlled by scrreprt for the initial stuff - look into this later

\maketitle

\tableofcontents
\listoffigures
\listoftables

\section{Overview}
\label{ch:overview}

The Serial Line Internet Protocol
(\href{https://tools.ietf.org/html/rfc1055}{RFC1055}), or SLIP,  is an
encapsulation of Internet Protocol (IP) datagrams for serial links. It provides
the bare minimum of encapsulation necessary to delimit the start and end of an
IP datagram on a serial link.

Because SLIP packets are simply thinly-encoded IP packets, routing is performed
using the standard IP address. The SLIP version of the Network Layer subsystem
interprets the SrcAddr register as the Host ID for the Elysium radio and
defines registers for both a Network ID and Subnet Mask for the Ground and
Spacecraft networks. From these components, a full IP address for the Elysium
on both the Ground and Spacecraft networks can be constructed and all valid
routing decisions can be made. More details of address translation and routing
can be found in \hyperref[sec:sliproute]{Section \ref{sec:sliproute}}.

The checksum protecting the IP packet header will be calculated for packets
sent to the Elysium firmware. Incorrect checksum values will result in the
packet being discarded. The checksum may optionally be checked on every packet,
even those which are routed by pass-through. This may have an effect on
throughput.

By default, the Elysium does not modify the TTL field of packets which are
routed by pass-through. The Elysium may optionally be configured to reduce the
TTL field by 1 when packets are routed through it. If both the checksum check
and the TTL reduction are configured, two checksum calculations per packet are
required - otherwise, a single checksum calculation per packet is required.
This may have an effect on throughput.

The Elysium implementation of the SLIP protocol can support packets larger than
the suggested 1006-byte packets mentioned in the RFC. The maximum packet size
can be set to between 40 and 9000 bytes at programming time or by modifying a
register value in-system.

Registers associated with the SLIP Network Layer subsystem can be found in
\hyperref[sec:slipregs]{Section \ref{sec:slipregs}}. Channels, Errors, and
Events associated with the SLIP Network Layer subsystem can be found in
\hyperref[sec:slipchan]{Section \ref{sec:slipchan}},
\hyperref[sec:sliperrs]{Section \ref{sec:sliperrs}}, and
\hyperref[sec:slipevts]{Section \ref{sec:slipevts}}, respectively.

\section{Routing}
\label{sec:sliproute}

In order to receive IP datagrams, the Elysium must have a valid IP address. In
fact, the Elysium has two valid IP addresses, one for the UART interface on the
Spacecraft network and one for the RF interface on the Ground network.
These addresses are derived from the values of the SrcAddr register, the
\hyperref[reg:scnetid]{SCNetID} and \hyperref[reg:gndnetid]{GndNetID}
registers, and the SCSubnet and GndSubnet fields of the
\hyperref[reg:subnets]{Subnets} register.

To construct the Elysium IP address on the Spacecraft network, the 32-bit value
of the SCNetID register is masked using the SCSubnet value to produce a Network
ID for the Spacecraft network. The 16-bit value of the SrcAddr register is also
masked using the SCSubnet value to produce a Host ID for the Elysium on the
Spacecraft network. The full Elysium IP address is then obtained by
concatenating the Network and Host IDs. For example, if the SCNetID is
192.168.0.0, the SCSubnet is /16, and the SrcAddr register is 0x6464
(0.0.100.100), then the Network ID becomes 192.168.0.0, the Host ID becomes
0.0.100.100, and the combined IP address becomes 192.168.100.100. To construct
the Ground network address, simply use the GndNetID and GndSubnet values. A
similar procedure can translate Reply Addresses into IP addresses.

Translations of IP addresses into Reply Addresses is performed by applying the
Subnet Mask of the source network to the IP address and taking the two
rightmost octects of the result.

Routing decisions are made by examining the Destination Address in the IP
header. If the Destination Address matches either of the Elysium IP addresses,
the header is removed and the data payload is routed to the Elysium firmware.
If the Destination Address falls within the Spacecraft network (its Network ID
matches the Spacecraft Network ID), the packet is routed out the UART interface
as a SLIP-encoded packet. If the Destination Address falls within the Ground
network, the packet is passed to the Data Link Layer for transmission over the
RF interface. Finally, if the Destination Address meets none of these criteria,
it is discarded and an error may be reported.

\section{Registers}
\label{sec:slipregs}
\setcounter{regaddr}{128}

This section defines the registers in \tabref{slipregs}, which apply to the
Serial Line Internet Protocol Network Layer.

\setcounter{regaddr}{128}
\begin{longtable}{Rcr}
		\caption{SLIP Registers}\\
		\label{tab:slipregs}\\
		\toprule
		\em Address & \em Name & \em Description\\
		\midrule
		\endhead\EndTableHeader
		\setlabref{reg:maxpktlength} & \labref{MaxPktLength0} & \labref{Maximum
			Packet Length LSB}\\
		\setlabref{reg:maxpktlength} & \labref{MaxPktLength1} & \labref{Maximum
			Packet Length MSB}\\
		\setlabref{reg:scnetid} & \labref{SCNetID0} & \labref{Spacecraft
			Network ID LSB}\\
		\setlabref{reg:scnetid} & \labref{SCNetID1} & \labref{Spacecraft
			Network ID Low Middle Byte}\\
		\setlabref{reg:scnetid} & \labref{SCNetID2} & \labref{Spacecraft
			Network ID High Middle Byte}\\
		\setlabref{reg:scnetid} & \labref{SCNetID3} & \labref{Spacecraft
			Network ID MSB}\\
		\setlabref{reg:gndnetid} & \labref{GndNetID0} & \labref{Ground
			Network ID LSB}\\
		\setlabref{reg:gndnetid} & \labref{GndNetID1} & \labref{Ground
			Network ID Low Middle Byte}\\
		\setlabref{reg:gndnetid} & \labref{GndNetID2} & \labref{Ground
			Network ID High Middle Byte}\\
		\setlabref{reg:gndnetid} & \labref{GndNetID3} & \labref{Ground
			Network ID MSB}\\
		\setlabref{reg:subnets} & \labref{Subnets} & \labref{Spacecraft and
			Ground network Subnets}\\
		\setlabref{reg:slipoptions} & \labref{Options} & \labref{General
			Configuration Bitfields}\\
		\setlabref{reg:elysiumport} & \labref{ElysiumPort0} & \labref{Elysium
			UDP Port LSB}\\
		\setlabref{reg:elysiumport} & \labref{ElysiumPort1} & \labref{Elysium
			UDP Port MSB}\\
		\setlabref{reg:checksumerrlvl} & \labref{ChecksumErrLvl} & 
			\labref{Invalid Header Checksum Error Reporting Level}\\
		\setlabref{reg:subneterrlvl} & \labref{SubnetErrLvl} & 
			\labref{Incorrect Subnet Error Reporting Level}\\
		\setlabref{reg:porterrlvl} & \labref{PortErrLvl} & 
			\labref{Incorrect Port Error Reporting Level}\\
		\setlabref{reg:mtuerrlvl} & \labref{MTUErrLvl} & 
			\labref{Datagram Too Big Error Reporting Level}\\
		\setlabref{reg:sliperrrpt} & \labref{SLIPErrRpt} & 
			\labref{SLIP Error Reporting Bitfields}\\
		\bottomrule
\end{longtable}
\setcounter{regaddr}{128}

\stoptocentries

\subsection{MaxPktLength[0-1]}

\noindent \textbf{Address:} \elyregaddr[2]

\noindent \textbf{Data Type:} uint16\_t

\noindent \textbf{Description:} The MaxPktLength register contains the maximum
length of an IP packet as a 16-bit unsigned integer in bytes.

\noindent \textbf{Diagram:}

\begin{register}{H}{MaxPktLength}{\curreg}
		\label{reg:maxpktlength}
		\regfield{MaxPktLength1}{8}{8}{0000_0101}
		\regfield{MaxPktLength0}{8}{0}{1101_1100}
		\reglabel{1500 Bytes}
\end{register}

\noindent \textbf{Fields:}

\begin{itemize}
		\item MaxPktLength1 - MSB - \curreg[1]
		\item MaxPktLength0 - LSB - \curreg[0]
\end{itemize}

\noindent \textbf{Recommended Value:} 1500 bytes (Ethernet MTU)

\noindent \textbf{Notes:} The valid range for this register is from 20 to 9000
bytes. 

The number of packets capable of being stored in the on-board buffer is
directly proportional to this value. The buffer is 9000 bytes deep and is
divided into slots of length MaxPktLength, with excess bytes remaining unused.

This value is also known as the Maximum Transmission Unit, or MTU, in IP
networking terminology.

The Elysium does not support fragmentation of IP datagrams which exceed the
MTU at this time.

\subsection{SCNetID[0-3]}

\noindent \textbf{Address:} \elyregaddr[4]

\noindent \textbf{Data Type:} uint32\_t

\noindent \textbf{Description:} The SCNetID register contains the Network ID
portion of the Elysium radio's spacecraft-facing IP interface. 

\noindent \textbf{Diagram:}

\begin{register}{H}{SCNetID}{\curreg}
		\label{reg:scnetid}
		\regfield{SCNetID3}{8}{32}{1100_0000}
		\regfield{SCNetID2}{8}{16}{1010_1000}
		\regfield{SCNetID1}{8}{8}{0000_0000}
		\regfield{SCNetID0}{8}{0}{0000_0000}
		\reglabel{192.168.0.0}
\end{register}

\noindent \textbf{Fields:}

\begin{itemize}
		\item SCNetID3 - MSB - \curreg[1]
		\item SCNetID2 - High middle byte - \curreg[0]
		\item SCNetID1 - Low middle byte - \curreg[1]
		\item SCNetID0 - LSB - \curreg[0]
\end{itemize}

\noindent \textbf{Recommended Value:} 192.168.0.0

\noindent \textbf{Notes:} This register contains only the Network ID of the
spacecraft-facing IP interface. The Network ID of the ground-facing (RF)
interface is configured in the \hyperref[reg:gndnetid]{GndNetID register} . The
Host ID of the Elysium (used on both networks) is configured in the SrcAddr
register. The \hyperref[reg:subnets]{Subnets register} contains both the ground
and space subnet masks to fully disambiguate the address.

\subsection{GndNetID}

\noindent \textbf{Address:} \elyregaddr[4]

\noindent \textbf{Data Type:} uint32\_t

\noindent \textbf{Description:} The GndNetID register contains the Network ID
portion of the Elysium radio's ground-facing IP interface. 

\noindent \textbf{Diagram:}

\begin{register}{H}{GndNetID}{\curreg}
		\label{reg:gndnetid}
		\regfield{GndNetID3}{8}{32}{1100_0000}
		\regfield{GndNetID2}{8}{16}{1010_1000}
		\regfield{GndNetID1}{8}{8}{0000_0000}
		\regfield{GndNetID0}{8}{0}{0000_0000}
		\reglabel{192.168.0.0}
\end{register}

\noindent \textbf{Fields:}

\begin{itemize}
		\item GndNetID3 - MSB - \curreg[1]
		\item GndNetID2 - High middle byte - \curreg[0]
		\item GndNetID1 - Low middle byte - \curreg[1]
		\item GndNetID0 - LSB - \curreg[0]
\end{itemize}

\noindent \textbf{Recommended Value:} 192.168.0.0

\noindent \textbf{Notes:} This register contains only the Network ID of the
ground-facing IP interface. The Network ID of the spacecraft-facing (UART)
interface is configured in the \hyperref[reg:scnetid]{SCNetID register} . The
Host ID of the Elysium (used on both networks) is configured in the
SrcAddr register. The
\hyperref[reg:subnets]{Subnets register} contains both the ground and space
subnet masks to fully disambiguate the address.

\subsection{Subnets}

\noindent \textbf{Address:} \elyregaddr

\noindent \textbf{Data Type:} uint8\_t

\noindent \textbf{Description:} The Subnets register contains the CIDR subnet
designators for the Elysiums ground- and spacecraft-facing IP interfaces.

\noindent \textbf{Diagram:}

\begin{register}{H}{Subnets}{\curreg}
		\label{reg:subnets}
		\regfield{SCSubnet}{4}{4}{0000}
		\regfield{GndSubnet}{4}{0}{0000}
		\reglabel{Both /16}
\end{register}

\noindent \textbf{Fields:}

\begin{itemize}
		\item SCSubnet - Spacecraft subnet CIDR designator - \curreg[0].4
		\item GndSubnet - Ground subnet CIDR designator - \curreg[0].0
\end{itemize}

\noindent \textbf{Recommended Value:} /16, /16

\noindent \textbf{Notes:} The two fields of this register encode CIDR subnet
designators for the ground- and spacecraft-facing IP interfaces. These
designators are encoded as the standard CIDR designator, minus 16. That is, a
value in the SCSubnet field of 0000 results in a CIDR designator of /16 for the
spacecraft link, while a value in the GndSubnet field of 1000 results in a CIDR
designator of /24 for the ground link. In this way we efficiently encode the
supported subnet masks of /16 through /31. Larger blocks, such as /12 or /8
subnets, are not supported by the Elysium.

This register contains the subnet masks for both the ground- and
spacecraft-facing IP interfaces.  The Network IDs of the ground-facing (RF) and
spacecraft-facing (UART) interfaces are configured in the
\hyperref[reg:gndnetid]{GndNetID register} and the
\hyperref[reg:scnetid]{SCNetID register}, respectively. The Host ID of the
Elysium (used on both networks) is configured in the
SrcAddr register.

\subsection{Options}

\noindent \textbf{Address:} \elyregaddr

\noindent \textbf{Data Type:} Bitfields

\noindent \textbf{Description:} The Options register contains a number of
bitfields which configure the use of optional features of the Serial Line
Internet Protocol.

\noindent \textbf{Diagram:}

\begin{register}{H}{Options}{\curreg}
		\label{reg:slipoptions}
		\regfield{Res.}{3}{5}{111}
		\regfield{TTL Decrease}{1}{4}{0}
		\regfield{Checksum All Headers}{1}{3}{0}
		\regfield{UDP}{1}{2}{0}
		\regfield{Header Compression}{2}{0}{00}
		\reglabel{No Options Enabled}
\end{register}

\noindent \textbf{Fields:}

\begin{itemize}
		\item Res. - Reserved. These bits are ignored. - \curreg.5
		\item TTL Decrease - TTL decrease enable field - \curreg.4
		\item Checksum All Headers - Calculate checksum for all packet headers 
				enable field - \curreg.3
		\item UDP - UDP enable field - \curreg.2
		\item Header Compression - Header Compression bitfield - \curreg.0
\end{itemize}

\noindent \textbf{Recommended Value:} Header compression (RoHC) should always
be used.  UDP should be used if it is required to interact with host software.

\noindent \textbf{Notes:} The reserved bits are ignored and may be safely set
to any value.

The TTL Decrease field, when set, causes the Elysium to decrease the TTL value
of every packet which is relayed between the two IP interfaces. This requires
that the checksum be recalculated as well.

The Checksum All Headers field, when set, causes the Elysium to calculate the
IPv4 header checksum of every packet which is relayed between the two IP
interfaces. Checksums are always calculated on packets addressed to the Elysium
firmware.

The UDP field, when set, causes the Elysium firmware to listen for commands on
a UDP port specified in the \hyperref[reg:elysiumport]{ElysiumPort register},
rather than interpreting the entire payload of all IP datagrams addressed to
the Elysium as commands. This port will also be set as the Source and
Destination ports of all packets sent by the Elysium firmware.

The Header Compression bitfield specifies the level and type of compression to
be performed on the headers of IP packets sent by the Elysium firmware, and to
be expected on the headers of IP packets set to the Elysium firmware. The
values of this bitfield are as shown below.

\begin{itemize}
		\item[00 -] Header compression is entirely disabled.
		\item[00 -] Header compression as specified in RFC 2507 (IPHC) is used.
		\item[01 -] Header compression as specified in RFC 2508 (CRTP) is used.
		\item[11 -] Header compression as specified in RFC 3095 (RoHC) is used.
\end{itemize}

The above list specifies the original standard for all types of header
compression - however, the most recent version of the standards is used.

\subsection{ElysiumPort[0-1]}

\noindent \textbf{Address:} \elyregaddr[2]

\noindent \textbf{Data Type:} uint16\_t

\noindent \textbf{Description:} The ElysiumPort register contains the UDP port
used by the Elysium firmware when the UDP bit of the
\hyperref[reg:slipoptions]{Options register} is set.

\noindent \textbf{Diagram:}

\begin{register}{H}{ElysiumPort}{\curreg}
		\label{reg:elysiumport}
		\regfield{ElysiumPort1}{8}{8}{1101_1111}
		\regfield{ElysiumPort0}{8}{0}{0000_0000}
		\reglabel{Port 57088}
\end{register}

\noindent \textbf{Fields:}

\begin{itemize}
		\item ElysiumPort1 - MSB - \curreg[1]
		\item ElysiumPort0 - LSB - \curreg[0]
\end{itemize}

\noindent \textbf{Recommended Value:} Any port in the range [49152-65535].

\noindent \textbf{Notes:} The ports in the range [49152-65535] are not reserved
or registered with IANA and are available for private use.

The Elysium will look for this port value in the Destination Port field of the
UDP header of incoming packets, and will set this port value as both the Source
and Destination Port fields of outgoing packets.

This register is only used if the UDP bit of the Options register is set.

\subsection{ChecksumErrLvl}

\noindent \textbf{Address:} \elyregaddr

\noindent \textbf{Data Type:} Priority Enumeration

\noindent \textbf{Description:} The ChecksumErrLvl register controls the
priority level of incorrect header checksum errors, if enabled by the
ChecksumErrRpt bit of the \hyperref[reg:sliperrrpt]{SLIPErrRpt register}.

\noindent \textbf{Diagram:}

\begin{register}{H}{ChecksumErrLvl}{\curreg}
		\label{reg:checksumerrlvl}
		\regfield{ChecksumErrLvl}{8}{0}{00001000}
		\reglabel{ERROR}
\end{register}

\noindent \textbf{Fields:}

\begin{itemize}
		\item ChecksumErrLvl - Priority level of incorrect checksum errors - 
				\curreg
\end{itemize}

\noindent \textbf{Recommended Value:} ERROR

\noindent \textbf{Notes:} The acceptable values for this register are the valid
values of the Priority Enumeration data type, a one-hot encoding using bits 0
through 4.

\subsection{SubnetErrLvl}

\noindent \textbf{Address:} \elyregaddr

\noindent \textbf{Data Type:} Priority Enumeration

\noindent \textbf{Description:} The SubnetErrLvl register controls the
priority level of incorrect subnet errors, if enabled by the
SubnetErrRpt bit of the \hyperref[reg:sliperrrpt]{SLIPErrRpt register}.

\noindent \textbf{Diagram:}

\begin{register}{H}{SubnetErrLvl}{\curreg}
		\label{reg:subneterrlvl}
		\regfield{SubnetErrLvl}{8}{0}{00001000}
		\reglabel{ERROR}
\end{register}

\noindent \textbf{Fields:}

\begin{itemize}
		\item SubnetErrLvl - Priority level of incorrect subnet errors - 
				\curreg
\end{itemize}

\noindent \textbf{Recommended Value:} ERROR

\noindent \textbf{Notes:} The acceptable values for this register are the valid
values of the Priority Enumeration data type, a one-hot encoding using bits 0
through 4.

\subsection{PortErrLvl}

\noindent \textbf{Address:} \elyregaddr

\noindent \textbf{Data Type:} Priority Enumeration

\noindent \textbf{Description:} The PortErrLvl register controls the
priority level of incorrect UDP port errors, if enabled by the
PortErrRpt bit of the \hyperref[reg:sliperrrpt]{SLIPErrRpt register}.

\noindent \textbf{Diagram:}

\begin{register}{H}{PortErrLvl}{\curreg}
		\label{reg:porterrlvl}
		\regfield{PortErrLvl}{8}{0}{00001000}
		\reglabel{ERROR}
\end{register}

\noindent \textbf{Fields:}

\begin{itemize}
		\item PortErrLvl - Priority level of incorrect UDP port errors - 
				\curreg
\end{itemize}

\noindent \textbf{Recommended Value:} ERROR

\noindent \textbf{Notes:} The acceptable values for this register are the valid
values of the Priority Enumeration data type, a one-hot encoding using bits 0
through 4.

\subsection{MTUErrLvl}

\noindent \textbf{Address:} \elyregaddr

\noindent \textbf{Data Type:} Priority Enumeration

\noindent \textbf{Description:} The MTUErrLvl register controls the
priority level of Datagram Too Big errors, if enabled by the
MTUErrRpt bit of the \hyperref[reg:sliperrrpt]{SLIPErrRpt register}.

\noindent \textbf{Diagram:}

\begin{register}{H}{MTUErrLvl}{\curreg}
		\label{reg:mtuerrlvl}
		\regfield{MTUErrLvl}{8}{0}{00001000}
		\reglabel{ERROR}
\end{register}

\noindent \textbf{Fields:}

\begin{itemize}
		\item MTUErrLvl - Priority level of Datagram Too Big errors - 
				\curreg
\end{itemize}

\noindent \textbf{Recommended Value:} ERROR

\noindent \textbf{Notes:} The acceptable values for this register are the valid
values of the Priority Enumeration data type, a one-hot encoding using bits 0
through 4.

\subsection{SLIPErrRpt}

\noindent \textbf{Address:} \elyregaddr

\noindent \textbf{Data Type:} Bitfields

\noindent \textbf{Description:} The SLIPErrRpt register contains a number of
bitfields controlling the reporting of errors within the SLIP Networking Layer.

\noindent \textbf{Diagram:}

\begin{register}{H}{SLIPErrRpt}{\curreg}
		\label{reg:sliperrrpt}
		\regfield{Res.}{4}{4}{1111}
		\regfield{ChecksumErrRpt}{1}{1}{1}
		\regfield{SubnetErrRpt}{1}{1}{1}
		\regfield{PortErrRpt}{1}{1}{1}
		\regfield{MTUErrRpt}{1}{0}{1}
		\reglabel{All Errors Reported}
\end{register}

\noindent \textbf{Fields:}

\begin{itemize}
		\item Res. - Reserved. These bits are ignored. - \curreg.4
		\item ChecksumErrRpt - Enables reporting of incorrect header checksum 
				errors - \curreg[0].3
		\item SubnetErrRpt - Enables reporting of incorrect subnet errors - 
				\curreg[0].2
		\item PortErrRpt - Enables reporting of incorrect UDP port errors - 
				\curreg[0].1
		\item MTUErrLvl - Enables reporting of Datagram Too Big errors -
				- \curreg[0].0
\end{itemize}

\noindent \textbf{Recommended Value:} If the spacecraft contains a flight
computer which is capable of taking action to correct errors, in general all
errors should be reported.

\noindent \textbf{Notes:} The reserved bits are ignored and may be safely set
to any value.

When the ChecksumErrRpt bit is set, anytime an IP packet is received by the
Elysium firmware (or is received for routing if the Checksum All Packets bit is
set in the \hyperref[reg:slipoptions]{Options register}) with an incorrect
header checksum, an error will be reported with the priority level defined in
the \hyperref[reg:checksumerrlvl]{ChecksumErrLvl register}.

When the SubnetErrRpt bit is set, anytime an IP packet is received for routing
with a Network ID not matching either the Ground or Spacecraft networks, an
error will be reported with the priority level defined in the
\hyperref[reg:subneterrlvl]{SubnetErrLvl register}.

When the PortErrRpt bit is set, anytime the UDP bit is set in the
\hyperref[reg:slipoptions]{Options register} and an IP packet is received by
the Elysium firmware with a UDP port not matching the
\hyperref[reg:elysiumport]{ElysiumPort register}, an error will be reported
with the priority level defined in the \hyperref[reg:subneterrlvl]{SubnetErrLvl
register}.

When the MTUErrRpt bit is set, anytime an IP packet is received for routing
with a Total Length field indicating that the packet is larger than the maximum
packet size specified in the \hyperref[reg:maxpktlength]{MaxPktLength
register}, an error will be reported with the priority level defined in the
\hyperref[reg:mtuerrlvl]{MTUErrLvl register}.

\starttocentries

\section{Channels}
\label{sec:slipchan}

\setcounter{idval}{96}
\begin{longtable}{Wcr}
		\caption{Channels}\\
		\label{tab:channels}\\
		\toprule
		\em ID & \em Name  & \em Data Type\\
		\midrule
		\endhead\EndTableHeader
		\setlabref{chan:slippktrecv} & \labref{Packets Received} & 
			\labref{uint8\_t}\\
		\setlabref{chan:slippktsend} & \labref{Packets Sent} & 
			\labref{uint16\_t}\\
		\setlabref{chan:slippktrelay} & \labref{Packets Relayed} & 
			\labref{uint8\_t}\\
		\bottomrule
\end{longtable}
\setcounter{idval}{96}

\stoptocentries

\subsection{Packets Receieved}
\label{chan:slippktrecv}

\noindent \textbf{Channel ID:} \elyid 

\noindent \textbf{Data Type:} uint8\_t

\noindent \textbf{Description:} The Packets Received Channel reports the number
of packets received by the Elysium firmware.

\noindent \textbf{Format:}
\newline
\newline
\begin{center}
\begin{bytefield}[endianness=big,bitwidth=2.25em]{8}
		\bitheader{0-7}\\
		\begin{rightwordgroup}{ID}
				\bitbox{8}{}
		\end{rightwordgroup}\\
		\begin{rightwordgroup}{Packets Received}
				\bitbox{8}{Value}
		\end{rightwordgroup}
\end{bytefield}
\end{center}

\noindent \textbf{Notes:} This counter increases monotonically until it rolls
over from 255 to 0.

\subsection{Packets Sent}
\label{chan:slippktsend}

\noindent \textbf{Channel ID:} \elyid 

\noindent \textbf{Data Type:} uint16\_t

\noindent \textbf{Description:} The Packets Sent Channel reports the number of
packets sent by the Elysium firmware.

\noindent \textbf{Format:}
\newline
\newline
\begin{center}
\begin{bytefield}[endianness=big,bitwidth=2.25em]{8}
		\bitheader{0-7}\\
		\begin{rightwordgroup}{ID}
				\bitbox{8}{}
		\end{rightwordgroup}\\
		\begin{rightwordgroup}{Packets Sent}
				\bitbox{8}{Value}
		\end{rightwordgroup}
\end{bytefield}
\end{center}

\noindent \textbf{Notes:} This counter increases monotonically until it rolls
over from 255 to 0.

\subsection{Packets Relayed}
\label{chan:slippktrelay}

\noindent \textbf{Channel ID:} \elyid 

\noindent \textbf{Data Type:} uint8\_t

\noindent \textbf{Description:} The Packets Relayed Channel reports the number
of packets relayed through the Elysium, without being received by or sent from
the Elysium firmware.

\noindent \textbf{Format:}
\newline
\newline
\begin{center}
\begin{bytefield}[endianness=big,bitwidth=2.25em]{8}
		\bitheader{0-7}\\
		\begin{rightwordgroup}{ID}
				\bitbox{8}{}
		\end{rightwordgroup}\\
		\begin{rightwordgroup}{Packets Relayed}
				\bitbox{8}{Value}
		\end{rightwordgroup}
\end{bytefield}
\end{center}

\noindent \textbf{Notes:} This value is incremented whenever a packet is passed
through the Elysium without interacting with the Elysium firmware, in either
direction.

This counter increases monotonically until it rolls over from 255 to 0.

\starttocentries

\begin{samepage}
\section{Errors}
\label{sec:sliperrs}

\setcounter{idval}{160}
\begin{longtable}[htbp]{Wr}
		\caption{Errors}\\
		\label{tab:errs}\\
		\toprule
		\em ID & \em Error\\
		\midrule
		\endhead\EndTableHeader
		\setlabref{err:slipsubnet} & \labref{Incorrect Subnet}\\
		\setlabref{err:slipmtu} & \labref{Datagram Too Big}\\
		\setlabref{err:slipchecksum} & \labref{Checksum Error}\\
		\setlabref{err:slipport} & \labref{Incorrect Port}\\
		\bottomrule
\end{longtable}
\setcounter{idval}{160}
\end{samepage}

\stoptocentries

\subsection{Incorrect Subnet}
\label{err:slipsubnet}

\noindent \textbf{Error ID:} \elyid 

\noindent \textbf{Description:} The Incorrect Subnet error indicates that a
packet to be routed has a Destination Address on an incorrect Subnet (i.e.,
neither the Spacecraft nor Ground subnets).

\noindent \textbf{Fault Response?} Packet discarded.

\noindent \textbf{Recommended Priority:} ERROR 

\noindent \textbf{Priority Register:} \hyperref[reg:subneterrlvl]{SubnetErrLvl}

\subsection{Datagram Too Big}
\label{err:slipmtu}

\noindent \textbf{Error ID:} \elyid 

\noindent \textbf{Description:} The Datagram Too Big error indicates that a
packet to be routed has a Total Length field which indicates a larger packet
than the MTU specified in the \hyperref[reg:maxpktlength]{MaxPktLength
register}.

\noindent \textbf{Fault Response?} Packet discarded.

\noindent \textbf{Recommended Priority:} ERROR 

\noindent \textbf{Priority Register:} \hyperref[reg:mtuerrlvl]{MTUErrLvl}

\subsection{Checksum Error}
\label{err:slipchecksum}

\noindent \textbf{Error ID:} \elyid 

\noindent \textbf{Description:} The Checksum Error error indicates that the
Elysium firmware has received a packet with an invalid Header Checksum, or that
the Checksum All Headers bit of the \hyperref[reg:slipoptions]{Options
register} is set and a packet to be routed has an invalid Header Checksum.

\noindent \textbf{Fault Response?} Packet discarded.

\noindent \textbf{Recommended Priority:} ERROR 

\noindent \textbf{Priority Register:}
\hyperref[reg:checksumerrlvl]{ChecksumErrLvl}

\subsection{Incorrect Port}
\label{err:slipport}

\noindent \textbf{Error ID:} \elyid 

\noindent \textbf{Description:} The Incorrect Port error indicates that the UDP
bit of the \hyperref[reg:slipoptions]{Options register} is set and the Elysium
firmware has received a packet with an incorrect Destination Port in its UDP
header.

\noindent \textbf{Fault Response?} Packet discarded.

\noindent \textbf{Recommended Priority:} ERROR 

\noindent \textbf{Priority Register:} \hyperref[reg:porterrlvl]{PortErrLvl}

\starttocentries

\section{Events}
\label{sec:slipevts}

\setcounter{idval}{224}
\begin{longtable}{Wr}
		\caption{Events}\\
		\label{tab:events}\\
		\toprule
		\em ID & \em Event\\
		\midrule
		\endhead\EndTableHeader
		\setlabref{evt:sliprecv} & \labref{Packet Received}\\
		\setlabref{evt:slipsent} & \labref{Packet Sent}\\
		\setlabref{evt:sliprelay} & \labref{Packet Relayed}\\
\end{longtable}
\setcounter{idval}{224}

\stoptocentries

\subsection{Packet Received}
\label{evt:sliprecv}

\noindent \textbf{Event ID:} \elyid 

\noindent \textbf{Description:} The Packet Received Event indicates that the
Elysium firmware has received a packet.

\noindent \textbf{Notes:} The total number of received packets can be retrieved
from the \hyperref[chan:slippktrecv]{Packets Received} channel.

\subsection{Packet Sent}
\label{evt:slipsent}

\noindent \textbf{Event ID:} \elyid 

\noindent \textbf{Description:} The Packet Transmitted Event indicates that the
Elysium firmware has sent out a packet.

\noindent \textbf{Notes:} The total number of sent packets can be retrieved
from the \hyperref[chan:slippktsend]{Packets Sent} channel.

\subsection{Packet Relayed}
\label{evt:sliprelay}

\noindent \textbf{Event ID:} \elyid 

\noindent \textbf{Description:} The Packet Relayed Event indicates that the
Elysium has relayed a packet.

\noindent \textbf{Notes:} The total number of relayed packets can be retrieved
from the \hyperref[chan:slippktrelay]{Packets Relayed} channel.

\starttocentries

\appendix

\section{Revision History}

\begin{enumerate}
		\item Initial release
\end{enumerate}

\end{document}
