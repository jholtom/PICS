\documentclass{hitec}

\author{Andrew Wygle}
\title{Elysium Radio Networking Layer - Space Packet Protocol}
\company{Adamant Aerospace}
\confidential{\textbf{Unlimited Distribution}}

\usepackage{graphicx}
\usepackage{booktabs}
\usepackage{pdfpages}
\usepackage{gensymb}
\usepackage{siunitx}
\usepackage{bytefield}
\usepackage[hyperref=true]{register}
\usepackage{rotating}
\usepackage{titlesec}
\usepackage{subcaption}
\usepackage{fmtcount}
\usepackage{parskip}
\usepackage{caption}
\usepackage[autostyle, english = american]{csquotes}
\usepackage{enumitem}
\usepackage{longtable}
\usepackage{array}
\usepackage[colorlinks=true, linktoc=all]{hyperref}
\usepackage{etoolbox}

\MakeOuterQuote{"}

\settextfraction{1}

\newcommand{\chref}[1]{\hyperref[ch:#1]{Section~\ref{ch:#1}}}
\newcommand{\figref}[1]{\hyperref[fig:#1]{Figure~\ref{fig:#1}}}
\newcommand{\tabref}[1]{\hyperref[tab:#1]{Table~\ref{tab:#1}}}
\makeatletter
\newcommand*{\textoverline}[1]{$\overline{\hbox{#1}}\m@th$}
\makeatother

\newcommand{\bitlabel}[2]{%
	\bitbox[]{#1}{%
		\raisebox{0pt}[4ex][0pt]{%
			\turnbox{45}{\fontsize{7}{7}\selectfont#2}%
		}%
	}%
}

\titleclass{\subsubsubsection}{straight}[\subsection]

\newcounter{subsubsubsection}[subsubsection]
\renewcommand\thesubsubsubsection{\thesubsubsection.\arabic{subsubsubsection}}
\renewcommand\theparagraph{\thesubsubsubsection.\arabic{paragraph}} % optional; useful if paragraphs are to be numbered

\titleformat{\subsubsubsection}
  {\normalfont\normalsize\bfseries}{\thesubsubsubsection}{1em}{}
\titlespacing*{\subsubsubsection}
{0pt}{3.25ex plus 1ex minus .2ex}{1.5ex plus .2ex}

\makeatletter
\renewcommand\paragraph{\@startsection{paragraph}{5}{\z@}%
  {3.25ex \@plus1ex \@minus.2ex}%
  {-1em}%
  {\normalfont\normalsize\bfseries}}
\renewcommand\subparagraph{\@startsection{subparagraph}{6}{\parindent}%
  {3.25ex \@plus1ex \@minus .2ex}%
  {-1em}%
  {\normalfont\normalsize\bfseries}}
\def\toclevel@subsubsubsection{4}
\def\toclevel@paragraph{5}
\def\toclevel@paragraph{6}
\def\l@subsubsubsection{\@dottedtocline{4}{7em}{4em}}
\def\l@paragraph{\@dottedtocline{5}{10em}{5em}}
\def\l@subparagraph{\@dottedtocline{6}{14em}{6em}}
\makeatother

\setcounter{secnumdepth}{4}
\setcounter{tocdepth}{4}

\titleformat{\subsubsection}
	{\normalfont\fontsize{12}{15}\bfseries}{\thesubsubsection}{1em}{}


\newcounter{regaddr}
\newcounter{regsize}
\newcommand*{\elyregaddr}[1][1]{0x\padzeroes[2]\Hexadecimal{regaddr}\addtocounter{regaddr}{#1}\setcounter{regsize}{#1}}

\newcommand*{\curreg}[1][0]{%
		\addtocounter{regaddr}{\the\numexpr#1-\value{regsize}\relax}%
		0x\padzeroes[2]\Hexadecimal{regaddr}%
		\addtocounter{regaddr}{\the\numexpr\value{regsize}-#1\relax}%
}

\newcounter{idval}
\newcommand*{\elyid}[0]{0x\padzeroes[2]\Hexadecimal{idval}\addtocounter{idval}{1}}

\newcommand*{\currid}[0]{%
		\addtocounter{idval}{-1}%
		0x\padzeroes[2]\Hexadecimal{idval}%
		\addtocounter{idval}{1}%
}

\newtoggle{inTableHeader}
\toggletrue{inTableHeader}
\newcommand*{\StartTableHeader}{\global\toggletrue{inTableHeader}}
\newcommand*{\EndTableHeader}{\global\togglefalse{inTableHeader}}
\let\OldLongtable\longtable%
\let\OldEndLongtable\endlongtable%
\renewenvironment{longtable}{\StartTableHeader\OldLongtable}{\OldEndLongtable\StartTableHeader}%

\newcommand*{\setlabref}[1]{\gdef\currlabref{#1}}
\newcommand*{\labref}[1]{\iftoggle{inTableHeader}{}{\hyperref[\currlabref]{#1}}}
\newcolumntype{W}{>{\global\let\currlabref\relax}l<{\labref{\elyid}}}
\newcolumntype{R}{>{\global\let\currlabref\relax}l<{\labref{\elyregaddr}}}

\let\oldaddcontentsline\addcontentsline
\newcommand{\stoptocentries}{\renewcommand{\addcontentsline}[3]{}}
\newcommand{\starttocentries}{\let\addcontentsline\oldaddcontentsline}

\begin{document}

% NOTE page numbering is controlled by scrreprt for the initial stuff - look into this later

\maketitle

\tableofcontents
\listoffigures
\listoftables

% Upon testing, we're going to use Chapters and Sections but not Parts
\section{Overview}
\label{ch:overview}

The Space Packet Protocol
(\href{https://public.ccsds.org/Pubs/133x0b1c2.pdf}{CCSDS Recommended Standard
133.0-B-1}) is a network layer standard used by many NASA and ESA missions.

Packets of the Space Packet Protocol are routed using Application Process
Identifiers (APIDs). The SrcAddr register will be interpreted as an APID
assigned to the Elyisum radio. All packets containing this APID will be routed
to the Elysium radio.

The Space Packet Protocol also makes a distinction between telemetry (or
reporting) packets and telecommand (or requesting) packets. With the exception
of packets containing the APID of the Elysium radio, all telemetry packets are
routed over the radio link, while all telecommand packets are routed over the
UART interface.

More details of address translation and routing can be found in
\hyperref[sec:spproute]{Section \ref{sec:spproute}}.

Registers associated with the SPP Network Layer subsystem can be found in
\hyperref[sec:sppregs]{Section \ref{sec:sppregs}}. Channels, Errors, and Events
associated with the SPP Network Layer subsystem can be found in
\hyperref[sec:sppchan]{Section \ref{sec:sppchan}},
\hyperref[sec:spperrs]{Section \ref{sec:spperrs}}, and
\hyperref[sec:sppevts]{Section \ref{sec:sppevts}}, respectively.

\section{Routing}
\label{sec:spproute}

In order to receive Space Packets, the Elysium must have a valid APID. 
This APID is derived from the value of the SrcAddr register by taking the least
significant 11 bits. Translations of APIDs into Reply Addresses is performed by
left-padding the APID into a 16-bit value.

Routing decisions are made by examining the Packet Type and APID of the Packet
Primary Header. If the Packet Type is Telecommand and the APID does not match
the Elysium APID, the packet is routed out the UART interface. If the Packet
Type is Telemetry and the source of the packet is not the Elysium, the packet
is passed to the Data Link Layer for transmission over the RF interface.

When Packet Type is Telemetry and the source of the packet is the Elysium,
the packet must be inspected further in
order to determine its destination. If the APID matches the value stored in the
\hyperref[reg:groundapid]{GroundAPID} register, the packet is passed to the
Data Link Layer for transmission over the RF interface - otherwise the packet
is routed out the UART interface.

As a small amount of framing is required to transmit Space Packets over the
UART link, the END and ESC bytes from the
\href{https://tools.ietf.org/html/rfc1055}{Serial Line Internet Protocol} are
used. An END byte (0xC0) indicates the end of a Space Packet (and thus the
start of a new one). An ESC byte is used to escape an END byte (with the
sequence 0xDB 0xDC) or an ESC byte (with the sequence 0xDB 0xDD) found in the
Packet.

\section{Registers}
\label{sec:sppregs}

This section defines the registers in \tabref{sppregs}, which apply to the
Space Packet Protocol Network Layer.

\setcounter{regaddr}{128}
\begin{longtable}{Rcr}
		\caption{SPP Registers}\\
		\label{tab:sppregs}\\
		\toprule
		\em Address & \em Name & \em Description\\
		\midrule
		\endhead\EndTableHeader
		\setlabref{reg:maxpktlength} & \labref{MaxPktLength0} & \labref{Maximum
			Packet Length LSB}\\
		\setlabref{reg:maxpktlength} & \labref{MaxPktLength1} & \labref{Maximum
			Packet Length MSB}\\
		\setlabref{reg:groundapid} & \labref{GroundAPID0} & \labref{Ground APID
			LSB}\\
		\setlabref{reg:groundapid} & \labref{GroundAPID1} & \labref{Ground APID
			MSB}\\
		\setlabref{reg:pktname} & \labref{PktName0} & \labref{Packet Name
			LSB}\\
		\setlabref{reg:pktname} & \labref{PktName1} & \labref{Packet Name
			MSB}\\
		\setlabref{reg:sppoptions} & \labref{Options} & \labref{General
			Configuration Bitfields}\\
		\setlabref{reg:pvnerrlvl} & \labref{PVNErrLvl} & 
			\labref{Invalid PVN Error Reporting Level}\\
		\setlabref{reg:pktlengthlvl} & \labref{PktLengthLvl} & 
			\labref{Packet Length Mismatch Error Reporting Level}\\
		\setlabref{reg:spperrrpt} & \labref{SPPErrRpt} & 
			\labref{SPP Error Reporting Bitfields}\\
		\bottomrule
\end{longtable}
\setcounter{regaddr}{128}

\stoptocentries

\subsection{MaxPktLength[0-1]}

\noindent \textbf{Address:} \elyregaddr[2]

\noindent \textbf{Data Type:} uint16\_t

\noindent \textbf{Description:} The MaxPktLength register contains the maximum
length of a Space Packet as a 16-bit unsigned integer in bytes. 

\noindent \textbf{Diagram:}

\begin{register}{H}{MaxPktLength}{\curreg}
		\label{reg:maxpktlength}
		\regfield{MaxPktLength1}{8}{8}{0001_0000}
		\regfield{MaxPktLength0}{8}{0}{0000_0000}
		\reglabel{4096 Bytes}
\end{register}

\noindent \textbf{Fields:}

\begin{itemize}
		\item MaxPktLength1 - MSB - \curreg[1]
		\item MaxPktLength0 - LSB - \curreg[0]
\end{itemize}

\noindent \textbf{Recommended Value:} 2048 bytes

\noindent \textbf{Notes:} The valid range for this register is from 7 to 4096
bytes. 

The number of packets capable of being stored in the on-board buffer is
directly proportional to this value. The buffer is 8192 bytes deep and is
divided into slots of length MaxPktLength, with excess bytes remaining unused.

\subsection{GroundAPID[0-1]}

\noindent \textbf{Address:} \elyregaddr[2]

\noindent \textbf{Data Type:} uint16\_t

\noindent \textbf{Description:} The GroundAPID register contains the APID used 
to identify the ground station when sending Telemetry packets from the Elysium
itself.

\noindent \textbf{Diagram:}

\begin{register}{H}{GroundAPID}{\curreg}
		\label{reg:groundapid}
		\regfield{GroundAPID1}{8}{8}{0000_0101}
		\regfield{GroundAPID0}{8}{0}{1001_1011}
		\reglabel{0x59B}
\end{register}

\noindent \textbf{Fields:}

\begin{itemize}
		\item GroundAPID1 - MSB - \curreg[1]
		\item GroundAPID0 - LSB - \curreg[0]
\end{itemize}

\noindent \textbf{Recommended Value:} N/A

\noindent \textbf{Notes:} The valid range for this register is from 0 to 2047
(0x000 to 0x7FF). A value over 0x7FF will be masked with 0x7FF to produce a
legal value.

This value will be used to determine when Elysium responses should be routed
over the radio to the ground station. Because all Elyisum response packets are
Telemetry packets, this APID is required to determine the routing destination
for such packets. All other packets are routed in a pass-through fashion as
described in \hyperref[sec:spproute]{Section \ref{sec:spproute}}.

\subsection{PktName[0-1]}

\noindent \textbf{Address:} \elyregaddr[2]

\noindent \textbf{Data Type:} uint16\_t

\noindent \textbf{Description:} The PktName register contains the Packet Name
used in place of the Packet Sequence Count if configured in the Packet Name
field of the \hyperref[reg:sppoptions]{Options register}.

\noindent \textbf{Diagram:}

\begin{register}{H}{PktName}{\curreg}
		\label{reg:pktname}
		\regfield{PktName1}{8}{8}{0001_1010}
		\regfield{PktName0}{8}{0}{0101_1100}
		\reglabel{0x1A5C}
\end{register}

\noindent \textbf{Fields:}

\begin{itemize}
		\item PktName1 - MSB - \curreg[1]
		\item PktName0 - LSB - \curreg[0]
\end{itemize}

\noindent \textbf{Recommended Value:} Use of this option is not recommended.

\noindent \textbf{Notes:} The valid range for this register is from 0 to 16383
(0x000 to 0x3FFF). A value over 0x3FFF will be clipped to 0x3FFF.

\subsection{Options}

\noindent \textbf{Address:} \elyregaddr

\noindent \textbf{Data Type:} Bitfields

\noindent \textbf{Description:} The Options register contains a number of
bitfields which configure the use of optional features of the Space Packet
Protocol.

\noindent \textbf{Diagram:}

\begin{register}{H}{Options}{\curreg}
		\label{reg:sppoptions}
		\regfield{Res.}{5}{3}{1111_1}
		\regfield{Packet Name}{1}{2}{0}
		\regfield{Timestamp}{1}{1}{0}
		\regfield{P-field}{1}{0}{0}
		\reglabel{No Options Enabled}
\end{register}

\noindent \textbf{Fields:}

\begin{itemize}
		\item Res. - Reserved. These bits are ignored. - \curreg.3
		\item Packet Name - Packet Name enable field - \curreg.2
		\item Timestamp - Timestamp enable field - \curreg.1
		\item P-Field - P-Field enable field - \curreg.1
\end{itemize}

\noindent \textbf{Recommended Value:} In general, none of these options are
needed, however if the Timestamp option is used the P-field option should also
be used.

\noindent \textbf{Notes:} The reserved bits are ignored and may be safely set
to any value.

The Packet Name field enables the use of the optional Packet Name function of
the Space Packet Protocol, which replaces the Packet Sequence Count with a
user-specified identifier (specified in the \hyperref[reg:pktname]{Packet Name 
register}).

The Timestamp field enables the use of the optional Time Code Field of the
Space Packet Protocol. The Time Code Field is an optional field in the Packet
Secondary Header which describes the time at which a packet was sent in one of
several formats. When this bit is set, all Telemetry packets sent by the
Elysium will contain in the Time Code Field the current Mission Time as a
32-bit signed integer. See the Elysium User Manual for more information.

The P-Field field enables the use of the optional P-Field portion of the Time
Code Field of the Space Packet Protocol. When this bit is set, a P-Field is
prepended to the 4-byte timestamp value to describe it as using an unsegmented
time code with an 
Agency-defined epoch and using 4 bytes for the basic time unit and 0 bytes
for the fractional time unit. This corresponds to a P-field value of 00101100
or 0x2C.

The Ancillary Data Field of the Space Packet Protocol is not supported by the
Elysium.

\subsection{PVNErrLvl}

\noindent \textbf{Address:} \elyregaddr

\noindent \textbf{Data Type:} Priority Enumeration

\noindent \textbf{Description:} The PVNErrLvl register controls the
priority level of invalid PVN errors, if enabled by the
PVNErrRpt bit of the \hyperref[reg:spperrrpt]{SPPErrRpt register}.

\noindent \textbf{Diagram:}

\begin{register}{H}{PVNErrLvl}{\curreg}
		\label{reg:pvnerrlvl}
		\regfield{PVNErrLvl}{8}{0}{00001000}
		\reglabel{ERROR}
\end{register}

\noindent \textbf{Fields:}

\begin{itemize}
		\item PVNErrLvl - Priority level of invalid PVN errors - \curreg
\end{itemize}

\noindent \textbf{Recommended Value:} ERROR

\noindent \textbf{Notes:} The acceptable values for this register are the valid
values of the Priority Enumeration data type, a one-hot encoding using bits 0
through 4.

\subsection{PktLengthLvl}

\noindent \textbf{Address:} \elyregaddr

\noindent \textbf{Data Type:} Priority Enumeration

\noindent \textbf{Description:} The PktLengthLvl register controls the
priority level of packet length mismatch errors, if enabled by the
PktLengthRpt bit of the \hyperref[reg:spperrrpt]{SPPErrRpt register}.

\noindent \textbf{Diagram:}

\begin{register}{H}{PktLengthLvl}{\curreg}
		\label{reg:pktlengthlvl}
		\regfield{PktLengthLvl}{8}{0}{00001000}
		\reglabel{ERROR}
\end{register}

\noindent \textbf{Fields:}

\begin{itemize}
		\item PktLengthLvl - Priority level of invalid PVN errors - \curreg
\end{itemize}

\noindent \textbf{Recommended Value:} ERROR

\noindent \textbf{Notes:} The acceptable values for this register are the valid
values of the Priority Enumeration data type, a one-hot encoding using bits 0
through 4.

\subsection{SPPErrRpt}

\noindent \textbf{Address:} \elyregaddr

\noindent \textbf{Data Type:} Bitfields

\noindent \textbf{Description:} The SPPErrRpt register contains a number of
bitfields controlling the reporting of errors within the SPP Networking Layer.

\noindent \textbf{Diagram:}

\begin{register}{H}{SPPErrRpt}{\curreg}
		\label{reg:spperrrpt}
		\regfield{Res.}{6}{2}{11_1111}
		\regfield{PVNErrRpt}{1}{1}{1}
		\regfield{PktLengthRpt}{1}{0}{1}
		\reglabel{All Errors Reported}
\end{register}

\noindent \textbf{Fields:}

\begin{itemize}
		\item Res. - Reserved. These bits are ignored. - \curreg.6
		\item PVNErrRpt - Enables reporting of invalid PVN errors - 
				\curreg[0].1
		\item PktLengthRpt - Enables reporting of packet length mismatch errors
				- \curreg[0].0
\end{itemize}

\noindent \textbf{Recommended Value:} If the spacecraft contains a flight
computer which is capable of taking action to correct errors, in general all
errors should be reported.

\noindent \textbf{Notes:} The reserved bits are ignored and may be safely set
to any value.

When the PVNErrRpt bit is set, anytime a Space Packet is received with an
invalid Packet Version Number, an error will be reported with the priority
level defined in the \hyperref[reg:pvnerrlvl]{PVNErrLvl register}.

When the PktLengthRpt bit is set, anytime a Space Packet is received which
consists of a different number of bytes than that specified in the Packet Data
Length field, an error will be reported with the priority
level defined in the \hyperref[reg:pktlengthlvl]{PktLengthLvl register}.

\starttocentries

\section{Channels}
\label{sec:sppchan}

\setcounter{idval}{96}
\begin{longtable}{Wcr}
		\caption{Channels}\\
		\label{tab:channels}\\
		\toprule
		\em ID & \em Name  & \em Data Type\\
		\midrule
		\endhead\EndTableHeader
		\setlabref{chan:spppktrecv} & \labref{Packets Received} & 
			\labref{uint8\_t}\\
		\setlabref{chan:spppktsend} & \labref{Packets Sent} & 
			\labref{uint16\_t}\\
		\setlabref{chan:spppktrelay} & \labref{Packets Relayed} & 
			\labref{uint8\_t}\\
		\bottomrule
\end{longtable}
\setcounter{idval}{96}

\stoptocentries

\subsection{Packets Receieved}
\label{chan:spppktrecv}

\noindent \textbf{Channel ID:} \elyid 

\noindent \textbf{Data Type:} uint8\_t

\noindent \textbf{Description:} The Packets Received Channel reports the number
of Telecommand packets received by the Elysium firmware.

\noindent \textbf{Format:}
\newline
\newline
\begin{center}
\begin{bytefield}[endianness=big,bitwidth=2.25em]{8}
		\bitheader{0-7}\\
		\begin{rightwordgroup}{ID}
				\bitbox{8}{}
		\end{rightwordgroup}\\
		\begin{rightwordgroup}{Packets Received}
				\bitbox{8}{Value}
		\end{rightwordgroup}
\end{bytefield}
\end{center}

\noindent \textbf{Notes:} This counter increases monotonically until it rolls
over from 255 to 0.

\subsection{Packets Sent}
\label{chan:spppktsend}

\noindent \textbf{Channel ID:} \elyid 

\noindent \textbf{Data Type:} uint16\_t

\noindent \textbf{Description:} The Packets Sent Channel reports the number of
Telemetry packets sent by the Elysium firmware.

\noindent \textbf{Format:}
\newline
\newline
\begin{center}
\begin{bytefield}[endianness=big,bitwidth=2.25em]{8}
		\bitheader{0-7}\\
		\begin{rightwordgroup}{ID}
				\bitbox{8}{}
		\end{rightwordgroup}\\
		\begin{rightwordgroup}{Packets Sent}
				\bitbox{8}{Value Byte 1}\\
				\bitbox{8}{Value Byte 0}
		\end{rightwordgroup}
\end{bytefield}
\end{center}

\noindent \textbf{Notes:} This value corresponds with the value to be placed in
the Packet Sequence Count field of the next Space Packet sent by the Elysium
(or the value that would be placed there if the Packet Name bit of the
\hyperref[reg:sppoptions]{Options register} were not set). As such, it is
limited to 14 bits, or the range from 0 to 16383.

This counter increases monotonically until it rolls over from 16383 to 0.

\subsection{Packets Relayed}
\label{chan:spppktrelay}

\noindent \textbf{Channel ID:} \elyid 

\noindent \textbf{Data Type:} uint8\_t

\noindent \textbf{Description:} The Packets Relayed Channel reports the number
of both Telemetry and Telecommand packets relayed through the Elysium, without
being received by or sent from the Elysium firmware.

\noindent \textbf{Format:}
\newline
\newline
\begin{center}
\begin{bytefield}[endianness=big,bitwidth=2.25em]{8}
		\bitheader{0-7}\\
		\begin{rightwordgroup}{ID}
				\bitbox{8}{}
		\end{rightwordgroup}\\
		\begin{rightwordgroup}{Packets Relayed}
				\bitbox{8}{Value}
		\end{rightwordgroup}
\end{bytefield}
\end{center}

\noindent \textbf{Notes:} This value is incremented whenever a packet is passed
through the Elysium without interacting with the Elysium firmware, in either
direction.

This counter increases monotonically until it rolls over from 255 to 0.

\starttocentries

\section{Errors}
\label{sec:spperrs}

\setcounter{idval}{160}
\begin{longtable}{Wr}
		\caption{Errors}\\
		\label{tab:errs}\\
		\toprule
		\em ID & \em Error\\
		\midrule
		\endhead\EndTableHeader
		\setlabref{err:spppvn} & \labref{PVN Mismatch}\\
		\setlabref{err:spplength} & \labref{Packet Length Mismatch}\\
		\bottomrule
\end{longtable}
\setcounter{idval}{160}

\stoptocentries

\subsection{PVN Mismatch}
\label{err:spppvn}

\noindent \textbf{Error ID:} \elyid 

\noindent \textbf{Description:} The PVN Mismatch error indicates that the
Elysium firwmare has received a Space Packet with a Packet Version Number (PVN)
other than '000'.

\noindent \textbf{Fault Response?} Packet discarded.

\noindent \textbf{Recommended Priority:} ERROR 

\noindent \textbf{Priority Register:} \hyperref[reg:pvnerrlvl]{PVNErrLvl}

\subsection{Packet Length Mismatch}
\label{err:spplength}

\noindent \textbf{Error ID:} \elyid 

\noindent \textbf{Description:} The Packet Length Mismatch error indicates that
the Elysium firwmare has received a Space Packet with a Packet Data Length
field which indicates either more or fewer bytes than the number of bytes
available in the packet.

\noindent \textbf{Fault Response?} Packet discarded.

\noindent \textbf{Recommended Priority:} ERROR 

\noindent \textbf{Priority Register:} \hyperref[reg:pktlengthlvl]{PktLengthLvl}

\starttocentries

\section{Events}
\label{sec:sppevts}

\setcounter{idval}{224}
\begin{longtable}{Wr}
		\caption{Events}\\
		\label{tab:events}\\
		\toprule
		\em ID & \em Event\\
		\midrule
		\endhead\EndTableHeader
		\setlabref{evt:spprecv} & \labref{Packet Received}\\
		\setlabref{evt:sppsent} & \labref{Packet Sent}\\
		\setlabref{evt:spprelay} & \labref{Packet Relayed}\\
\end{longtable}
\setcounter{idval}{224}

\stoptocentries

\subsection{Packet Received}
\label{evt:spprecv}

\noindent \textbf{Event ID:} \elyid 

\noindent \textbf{Description:} The Packet Received Event indicates that the
Elysium firmware has received a Telecommand packet.

\noindent \textbf{Notes:} The total number of received packets can be retrieved
from the \hyperref[chan:spppktrecv]{Packets Received} channel.

\subsection{Packet Sent}
\label{evt:sppsent}

\noindent \textbf{Event ID:} \elyid 

\noindent \textbf{Description:} The Packet Transmitted Event indicates that the
Elysium firmware has sent out a Telemetry packet.

\noindent \textbf{Notes:} The total number of sent packets can be retrieved
from the \hyperref[chan:spppktsend]{Packets Sent} channel.

\subsection{Packet Relayed}
\label{evt:spprelay}

\noindent \textbf{Event ID:} \elyid 

\noindent \textbf{Description:} The Packet Relayed Event indicates that the
Elysium has relayed a Space Packet.

\noindent \textbf{Notes:} The total number of relayed packets can be retrieved
from the \hyperref[chan:spppktrelay]{Packets Relayed} channel.

\starttocentries

\appendix

\section{Revision History}

\begin{enumerate}
		\item Initial release
\end{enumerate}

\end{document}
